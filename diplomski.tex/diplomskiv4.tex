\documentclass[a4paper, 12pt]{report}
\usepackage{subcaption}
\usepackage[croatian]{babel}
\usepackage[utf8]{inputenc}
\usepackage[T1]{fontenc}
\usepackage{graphicx}
\usepackage{titletoc}
\usepackage{longtable}
\usepackage[page, toc]{appendix}
\usepackage{arydshln}
\usepackage{afterpage}
\usepackage{microtype}
\usepackage[euler]{textgreek}
\usepackage{booktabs}
\usepackage[autostyle, style=croatian]{csquotes}
\usepackage[left=3cm, right=3cm, top=3cm, bottom=3cm]{geometry}
\usepackage[onehalfspacing]{setspace}
\usepackage{titlesec}
\usepackage{multirow}
\usepackage[natbibapa]{apacite}
\usepackage{enumitem}
\usepackage{lmodern}
\usepackage[dvipsnames]{xcolor}
\usepackage{caption}
\usepackage{floatrow}
\usepackage{lscape}
\usepackage{amsmath}
\usepackage{url}

\setlength{\dimen\footins}{5cm}

\renewcommand\appendixtocname{Dodaci}
\addcontentsline{toc}{section}{References}

\DeclareFloatFont{cstm}{\fontsize{11}{13}\selectfont}
\floatsetup[table]{font=cstm, style = plaintop, captionskip = 3pt}

\DeclareCaptionFormat{myformat}{\fontsize{11}{13}\selectfont #1#2\\#3}

\captionsetup*[figure]{labelsep = period, labelfont = it, justification =
    justified, position = bottom, font = {singlespacing, cstm}}
\captionsetup*[table]{labelsep = none, labelfont = it, justification =
    centering, font = {singlespacing}, format = myformat}
\captionsetup*[subfigure]{labelsep = none, labelfont = it, justification =
    centering, font = {singlespacing}, format = myformat, labelformat=empty}

\definecolor{zelena}{HTML}{20e555}
\definecolor{plava}{HTML}{4C4CFF}

\renewcommand\appendixpagename{Dodaci}

\newcommand{\periodafter}[1]{#1.}

\renewcommand{\rmdefault}{ptm}

\addto\captionscroatian{\renewcommand{\bibname}{\Large\bfseries Literatura}}

\renewcommand{\BCBT}{}%  comma between authors in ref. list when no. of
                      %%  authors = 2
\renewcommand{\BCBL}{}%  comma before last author when no. of authors > 2
\renewcommand{\BOthers}[1]{i sur.\hbox{}}% ``and others''
\renewcommand{\BBAA}{i}
\renewcommand{\BBAB}{i}
\renewcommand{\BIn}{U:}
\renewcommand{\BED}{Ur.\hbox{}}
\renewcommand{\BEDS}{Ur.\hbox{}}
\renewcommand{\BPGS}{str.\hbox{}}
\renewcommand{\BRetrievedFrom}{Preuzeto s: }

\titleformat{\section}{\Large\bfseries}{}{0pt}{}
\titleformat{\subsection}{\large\itshape}{}{0pt}{}
\titleformat{\paragraph}[runin]{\normalsize\itshape}{}{0pt}{\periodafter}

\titlespacing{\paragraph}{%
    \parindent}{%              left margin
  0.1\baselineskip}{% space before (vertical)
  0.5em}%               space after (horizontal)

\setlength{\bibhang}{1cm}
\setlength{\bibsep}{13pt}

\newcommand{\nocontentsline}[3]{}
\newcommand{\tocless}[2]{\bgroup\let\addcontentsline=\nocontentsline#1{#2}\egroup}

\begin{document}
\begin{titlepage}
    \begin{center}
{Sveučilište u Zagrebu\\
    Filozofski fakultet\\
    Odsjek za psihologiju\\}
\vspace{9.5cm}
{\bfseries TEST IMPLICITNIH ASOCIJACIJA ZA AGRESIVNOST KAO PREDIKTOR
    ODLUKA U EKSPERIMENTALNIM IGRAMA\\}
    Diplomski rad\\
\vspace{5cm}
Denis Vlašiček\\[0.7cm]
Mentor: izv. prof. dr. sc. Zvonimir Galić\\
\vspace{\fill}
Zagreb, 2017.
    \end{center}
\end{titlepage}

\setcounter{secnumdepth}{0} 
\tableofcontents
\thispagestyle{empty}

\begin{singlespace}
    \tocless\section{Sažetak}

Cilj istraživanja bio je proširiti spoznaje vezane uz bihevioralnu teoriju igara i
implicitne mjere ličnosti, konkretno, uz
test implicitnih asocijacija za agresivnost. Istraživanje je provedeno na 109
sudionika, većinom studenata. Sudionici su rješavali Buss-Perry upitnik
agresivnosti (BPAQ) i test implicitnih asocijacija za agresivnost (IAT-A) te
sudjelovali u Igri kukavice, Dilemi zatvorenika i Igri vođe. Rečeno im je da
igraju protiv anonimnog suigrača, kojeg u stvari nije bilo, nego su igrali
protiv računala s unaprijed određenim strategijama (surađuje u sve tri igre ili
ni u jednoj). Očekivali smo da će na temelju korištenih mjera, njihove međusobne
interakcije te interakcije s ponašanjem suigrača biti moguće predvidjeti
donošenje nekooperativnih odluka u eksperimentalnim igrama. Same mjere i njihova
dvostruka interakcija te dvostruka interakcija s ponašanjem suigrača nisu bili
značajni prediktori. Trostruke interakcije korištenih mjera s položajem igre u
slijedu i s ponašanjem suigrača pokazale su se prediktivnima za donošenje
nekooperativnih odluka u Dilemi zatvorenika odnosno u trećoj igri po redu.
Iako metodološka ograničenja nalažu da se rezultati uzmu s oprezom, smatramo da 
ovi nalazi mogu biti koristan orijentir za potencijalna nova istraživanja,
posebice zbog razmjerno malog broja istraživanja koja se bave ulogom
agresivnosti u eksperimentalnim igrama.
\bigskip

\noindent Ključne riječi: test implicitnih asocijacija, agresivnost, 
bihevioralna teorija igara, Igra kukavice, Igra vođe, Dilema zatvorenika, provokacija.

\vspace*{\fill}

\tocless\section{Summary}

\noindent The aim of this research was to expand upon the knowledge of behavioral game theory
and implicit personality measures, specifically, the implicit associations test
for aggression. The study was conducted on 109 participants, mostly students.
The participants completed the Buss-Perry Aggression Questionnaire (BPAQ) and
the implicit associations test for aggression (IAT-A), and played the Game of
chicken, Prisoner's Dilemma and Leader. The participants were told that they are
playing against an anonymous opponent, which, in reality, did not exist. Their
opponent was a computer with preprogrammed strategies (always cooperate or
always defect). We assumed that it would be possible to predict defecting
choices based on the measures used, their mutual interaction and their
interaction with the opponent's behavior. Neither the measures alone nor their
mutual two-way interaction or the two-way interaction with the opponent's
behavior were statistically significant predictors. The three-way interactions
of the measures with the game's position in the series and with the opponent's
behavior were predictive of defecting choices in the Prisoner's Dilemma and the
third game in the series, respectively. Although the conclusions of this study
should be taken with caution due to the methodological limitations, we believe
they can be a useful guidance for future research, especially because of the
relatively low number of studies dealing with the role of aggression in
experimental games.
\bigskip

\noindent Keywords: implicit associations test, aggression, behavioral game
theory, Game of Chicken, Prisoner's Dilemma, Leader, provocation.
\end{singlespace}

\thispagestyle{empty}
\clearpage

\setcounter{page}{1}
\section{Uvod}

Ovaj rad predstavlja pokušaj integriranja spoznaja o ponašanju u
eksperimentalnim igrama i spoznaja o implicitnoj agresivnosti.
Eksperimentalnim igrama smatramo sve situacije u kojima: 
\begin{enumerate}[label = (\alph*)]
    \item postoje barem dvije osobe (\emph{igrači}) koje donose odluke;
    \item svaka osoba može birati između dvije ili više opcija
        (\emph{strategija}), pri čemu ishod
        interakcije ovisi o odlukama svih sudionika;
    \item svi sudionici imaju dobro definirane preferencije  različitih
        ishoda, pri čemu se brojčane vrijednosti (\emph{dobiti}) koje odražavaju te
        preferencije mogu pridijeliti svim igračima za sve ishode \citep{colgt}.
\end{enumerate}

Kad govorimo o agresivnosti, možemo razlikovati instrumentalnu i hostilnu agresivnost, što
možemo povezati s eksplicitnim i implicitnim procesima obrade informacija, 
odnosno eksplicitnim i implicitnim komponentama ličnosti \citep{richetin2008automatic}.
Odlučili smo ispitivati agresivnost jer smatramo da je to osobina koja bi mogla
ležati u podlozi brojnih oblika nekooperativnog ponašanja u svakodnevnom životu.
Prvo ćemo pobliže objasniti teoriju igara i spoznaje vezane uz nju, a potom se
osvrnuti na implicitne i eksplicitne komponente ličnosti, posebice agresivnosti.

\subsection{Teorija igara}

Svaki dan ulazimo u interakciju sa stotinama ljudi. To mogu biti drugi vozači u
gustom prometu tijekom prometne špice ili poslodavac s kojim pregovaramo o
višoj plaći. James Surowiecki \citeyearpar{wisdcrowd} u svojoj knjizi
\emph{Mudrost masa} navodi da te dvije situacije možemo svrstati u probleme
koordinacije odnosno u probleme kooperacije. 
Problemi koordinacije odnose se na situacije u kojima skupina pojedinaca
pokušava uskladiti svoje ponašanje s ponašanjem drugih, znajući da svi drugi
pokušavaju učiniti isto. S druge strane, problemi kooperacije
opisuju situacije u kojima je potrebno skupinu ljudi koji gledaju samo na 
osobni interes navesti da surađuju, unatoč tome što osobni interes sugerira da
nitko ne bi trebao surađivati. 
Širok spektar problema koje bismo mogli klasificirati u skladu sa
Surowieckijevom \citeyearpar{wisdcrowd} podjelom već desetljećima proučava
teorija igara --- područje matematike koje se bavi logikom donošenja odluka u
socijalnim interakcijama \citep{colgt}. 

Teorija igara bavi se utvrđivanjem strategija koje 
predstavljaju najbolji odgovor 
pojedinog igrača  na strategije koje su odabrali drugi igrači 
\citep{mccain2010g, colgt}. Pritom, najboljom strategijom smatramo onu 
koja igraču osigurava najveću dobit s obzirom na strategiju koju su suigrači odabrali
ili za koju se može očekivati da će je odabrati \citep{mccain2010g}. 
\citet{colgt} navodi kako teorija igara problemu određivanja najbolje
strategije pristupa isključivo putem formalnog rezoniranja, utvrđujući 
 strategije koje bi igrači trebali koristiti kako bi racionalno slijedili osobne interese.
Teorija igara je, stoga, normativna, budući da nastoji otkriti kako bi se igrači
\emph{trebali} ponašati, ne stvarajući pritom predviđanja o tome kako će se oni
\emph{zapravo} ponašati \citep{colgt}. 

Proučavanje igara je značajno jer one predstavljaju jednostavne 
modele različitih međuljudskih interakcija
---  mnogi ekonomski, politički, vojni i
interpersonalni sukobi imaju  njihove karakteristike  \citep{worldexgame,colgt}. \citet[str. xiii]{gintbounds} čak
tvrdi da je teorija igara \enquote{[...] centralna za razumijevanje dinamike
    živih bića općenito, a posebice ljudi.} S obzirom na velik značaj teorije
igara u proučavanju donošenja odluka, ne iznenađuje ogroman korpus radova koji
se bave tom temom, kao ni širenje teorije igara van granica matematike, u
psihologiju, sociologiju, ekonomiju, političke znanosti i druge discipline
\citep{vlangsocrev}. Također, ne iznenađuje niti to da se javila želja za
odmicanjem od izrazito matematičkog i introspektivnog pristupa igrama ka
proučavanju \emph{stvarnog} ponašanja igrača --- ka onome što je poznato kao bihevioralna
teorija igara \citep{camerer2003}. Njenu važnost opisuje 
Gintisova \citeyearpar[str. xiii]{gintbounds}  tvrdnja da
\enquote{[...] kako je teorija igara bez šire socijalne teorije samo tehničko
    razmetanje, tako je i socijalna teorija bez teorije igara hendikepiran
    pothvat}, s čime se slaže i \citet[str. ix, par. 2]{colgt}. 

Eksperimentalne igre  snažan su alat u istraživanju ljudskog donošenja odluka. U
dobro dizajniranoj igri, odluke igrača odražavaju njihove motive, vrijednosti
i strategije \citep{worldexgame}. Odluke u eksperimentalnim igrama predstavljaju
uzorke ponašanja --- sudionici ne navode samo kako se osjećaju ili koji su
njihovi stavovi, nego donose odluke koje mogu imati stvarne posljedice za njih
same, ali i za druge \citep{worldexgame}. Ipak, eksperimentalne igre imaju i
neke nedostatke.
Jedan od najvećih problema vezanih uz eksperimentalne igre kao kriterijske mjere
u psihologijskim istraživanjima je višeznačnost donesenih odluka.
Primjerice, nesuradnja u Dilemi
zatvorenika može ukazivati na pohlepu, averziju prema gubicima ili defenzivnost
\citep{worldexgame, colgt}. Ipak, budući da je ponašanje ljudi u stvarnim
situacijama multifaktorski određeno, ovaj \enquote{nedostatak} može doprinijeti povećanju
ekološke valjanosti igara.

Dilema zatvorenika jedna je od najpoznatijih i najčešće istraživanih
eksperimentalnih igara \citep{colgt,
    van1998psychology, vancoop}. Scenarij
igre je ovakav: dvije osobe uhićene su i optužene za počinjenje kaznenog djela.
Nalaze se u policijskoj postaji na odvojenom ispitivanju. Tužitelj svakom od
njih ponudi da svjedoči protiv drugoga. Ako obojica odluče priznati zločin i
svjedočiti protiv drugoga, svaki će dobiti, primjerice, 3 godine zatvora. Ako
obojica odluče šutjeti, svaki će dobiti jednu godinu zatvora. Ali ako jedan
odluči svjedočiti, a drugi ne, onaj koji je priznao zločin bit će pušten na slobodu, dok će
onaj drugi dobiti kaznu zatvora u trajanju od 10 godina.

Dominantna strategija u ovoj igri --- ona koja dovodi do boljeg ishoda za
pojedinca, neovisno o tome koju strategiju njegov suigrač odabere --- je
nesuradnja \citep{colgt}. Stoga, teoretski modeli pretpostavljaju da će sudionici u
Dilemi zatvorenika --- kao racionalni agenti --- odbiti suradnju, budući da time
izbjegavaju najgori mogući ishod \citep{worldexgame}. Naime, ako pojedinac
odluči surađivati, a njegov suigrač odluči ne surađivati, pojedinac će dobiti
najvišu moguću zatvorsku kaznu, dok će njegov suigrač biti pušten na slobodu. S
druge strane, ako pojedinac ne surađuje, onda će, (a) u slučaju da njegov suigrač
surađuje, postići najbolji mogući ishod, a (b) ako suigrač ne surađuje, zatvorska
kazna koju će pojedinac dobiti neće biti maksimalna. Stoga, nesuradnja je
strategija koja maksimizira najmanju moguću dobit, što je čini strategijom koju
bi trebao odabrati svaki racionalan igrač. 
Rezultati istraživanja,
pak, pokazuju da je postotak suradničkih odluka viši od očekivanog \citep{colgt,
    worldexgame, jones2008smarter, shafir1992thinking}.

Gledajući razliku između očekivanog i realnog ponašanja, možemo se zapitati o
uzrocima te diskrepancije. Kroz desetljeća rada, istraživači su otkrili neke
determinante donošenja odluka u eksperimentalnim igrama. 
\Citet{vlangsocrev} sažimaju neke glavne nalaze, dijeleći ih pritom --- u skladu
s  okvirom teorije međuzavisnosti Kelleya i Thibauta \citeyearpar{kelthibinterp} --- na
spoznaje vezane uz strukturalne, psihološke i dinamičke utjecaje.
Ovdje će fokus biti na psihološkim utjecajima, budući da su oni
najrelevantniji za ovaj rad.

\subsection{Psihologijske odrednice ponašanja u eksperimentalnim igrama}

\textit{Socijalna vrijednosna orijentacija.} Varijabla koja je dobila velik
dio pažnje je socijalna vrijednosna orijentacija (SVO)
\citep{messicksov,vanpursuit} --- kronična preferencija određene distribucije
ishoda između sebe i zavisnih drugih \citep{handdedreu}. \citet{krugsochand}
govore o šest orijentacija, koje
kategoriziraju u tri skupine: prosocijalne  (suradnja, jednakost i
altruizam), sebične (individualizam i
natjecanje) te antisocijalne orijentacije (agresivnost). 

Pojedinci s prosocijalnom orijentacijom teže skladu i poštenju te cijene
kolektivnu dobrobit, dok oni sa sebičnom orijentacijom traže dobre ishode za
sebe, zanemarujući pritom ishode za druge \citep{handdedreu}. Nadalje,
pokazalo se da osobe s prosocijalnom orijentacijom veliku važnost pridaju
jednakosti ishoda --- reagiraju ljutnjom na kršenje jednakosti, neovisno o tome
kako to utječe na ishode za njih, dok osobe sebične orijentacije na kršenje
jednakosti reagiraju tek kad to šteti njihovim ishodima \citep{vancoop}.
Naposljetku, pokazalo se da prosocijalno orijentirani pojedinci vjeruju da
kooperativno ponašanje odražava inteligenciju te da je to ispravan način
ponašanja, dok sebično orijentirani pojedinci surađivanje smatraju znakom
slabosti
\citep*{liebrand1986might,mcclintock1988role,langelieb91,van1990causal}.

Socijalna vrijednosna orijentacija koja je najzanimljivija za naše istraživanje
je agresivnost, ali tijekom pretraživanja literature nismo pronašli radove koji
se tom temom bave u ovom kontekstu. 
\Citet{vancoop} osvrću se na nedostatak istraživanja vezanih uz
agresivnost te ističu važnost tog smjera istraživanja.

\paragraph{Povjerenje} Prema jednoj od definicija, povjerenje je
\enquote{psihološko stanje koje proizlazi iz namjere da se prihvati ranjivost
    na temelju pozitivnih očekivanja o namjerama ili ponašanjima drugih}
\citep*[str. 395]{rousseau1998not}.
\citet{20yexg} smatraju da je za suradnju važno očekivanje da će druga
strana surađivati jer se osoba u protivnom izlaže opasnosti od izrabljivanja.
Pokazalo se da osobe koje ne vjeruju drugima nisu nužno nekooperativne, ali
su sklone vjerovati da drugi neće surađivati, što umanjuje njihovu spremnost na
suradnju \citep{vancoop}. \citet{balliet2013trust}
to potvrđuju meta-analizom, u kojoj su pokazali da postoji snažna pozitivna korelacija između očekivanja o
ponašanju drugih i vlastitog surađivanja ($r =$ 0.58). S druge strane, ista
meta-analiza pokazuje da je korelacija
dispozicijskog povjerenja i suradnje znatno  niža ($r =$ 0.26). Nadalje, čini se da je povjerenje važnije kad ljudima
nedostaju informacije o namjerama ili ponašanjima drugih, ili kad se suočavaju s
višim razinama neizvjesnosti \citep{vancoop}.  

\paragraph{Atribucije} Atribucije su naši zaključci o uzrocima ponašanja drugih
osoba \citep{aronson}. \citet{kelley1970inference} su pokazali da
nekooperativni pojedinci često pogrešno prosuđuju namjere o suradnji
drugih ljudi, odnosno, da ih češće no što je opravdano smatraju nekooperativnima. Osim
toga, pokazali su da sudionici koji preferiraju kompetitivne strategije vjeruju
da su drugi ljudi uglavnom kompetitivni, dok kooperativni pojedinci smatraju
da su neki ljudi kompetitivni, a neki kooperativni \citep{kelley1970social}.

\paragraph{Ostale individualne razlike} \Citet{vancoop} upućuju na 
istraživanja koja su proučavala odnos individualnih
razlika i odluka u igrama. Između ostalog, navode da je suradnja viša kod osoba koje su nisko na
narcizmu i dispozicijskoj zavisti, nisko na ekstraverziji i visoko na ugodnosti
te kod onih koji su visoko na intrinzičnoj vrijednosnoj orijentaciji. Ipak,
pretraživanjem literature dobiva se dojam da manjka radova koji se bave
konstruktima koji su široko zastupljeni u raznim granama psihologije,
kao što su samopoimanje, razni modeli ličnosti ili inteligencija. 
\citet[str. 290, par. 1]{koole2001social} --- koji su
utvrdili prije spomenutu povezanost ekstraverzije i ugodnosti s donošenjem
odluka --- navode da je neuspjeh u pronalaženju konzistentnog utjecaja ličnosti
neke istraživače naveo da sumnjaju u značaj varijabli ličnosti kao značajnih
determinanti ponašanja u eksperimentalnim igrama. Možda se neočekivano mali broj istraživanja
uistinu može pripisati nekonzistentnosti nalaza. Ipak, smatramo da je
nekonzistentnost upravo razlog \emph{za} detaljnije ispitivanje tih konstrukata.
Cilj ovog istraživanja je obogatiti korpus literature vezane uz donošenje odluka
u eksperimentalnim igrama
i osobine ličnosti --- konkretno, u fokusu su implicitne i eksplicitne
mjere agresivnosti. Detaljna elaboracija odabira upravo agresivnosti kao
prediktora nalazi se u dijelu \emph{Naše istraživanje}.

\subsection{Implicitne i eksplicitne mjere ličnosti}

Mjere samoprocjene, odnosno, eksplicitne mjere često se koriste u
istraživanju ličnosti \citep{wileyhandzgal}. One zahvaćaju aspekte ponašanja,
kognicije, emocija, vrijednosti i
stavova kojih su ljudi svjesni \citep{lebretsubclin}. 
Unatoč njihovoj čestoj uporabi i
važnosti za opisivanje ličnosti pojedinca, one imaju niz problema. 
Činjenica da zahvaćaju samo aspekte kojih su ljudi svjesni ukazuje na to da
propušta one kojih nisu, što je značajan propust s obzirom na to da brojni
fenomeni djeluju van kompletne svijesti i kontrole te da ljudi nekih svojih
osobina nisu u potpunosti svjesni \citep{uhlmann2012getting, wileyhandzgal, aamondt2009org}.
Osim toga, budući da zahvaćaju aspekte kojih su ljudi svjesni, postoji opravdana
sumnja u točnost informacija --- ljudi mogu zatajiti informacije ili ih namjerno
iskriviti \citep{wileyhandzgal}. 

Implicitne mjere --- koje zahvaćaju automatske i nesvjesne procese --- zaobilaze
gore navedene poteškoće \citep{uhlmann2012getting, wileyhandzgal}. Osim što se
efikasno nose s nekim nedostacima eksplicitnih mjera, istraživanja pokazuju da
implicitne mjere imaju inkrementalnu valjanost nad eksplicitnima u objašnjavanju
različitih kriterija \citep{greenwald2009meta, banse2015predicting,
    galic2016conditional, galic2014validity}.
Ipak, implicitne  mjere ne bi trebale zamijeniti eksplicitne, budući da obje
vrste mjera zahvaćaju zasebne aspekte ličnosti \citep{bing2007integrating}. Osim
toga, i predviđaju različite ishode --- \citet{mcclelland1989self} se
referiraju na McClellandove (1980) dokaze da implicitni motivi predviđaju
spontane ponašajne trendove tijekom vremena, dok eksplicitni motivi predviđaju
trenutne, specifične reakcije na specifične situacije.

\subsection{Test implicitnih asocijacija kao mjera agresivnosti}Fizičko ili verbalno ponašanje koje se
očituje u neprijateljstvu
prema osobama ili predmetima i namjeri da im se nanese šteta nazivamo
agresivnošću \citep{rjecnik}.
Možemo razlikovati agresivnost koja je potaknuta srdžbom i čiji je cilj nanijeti
štetu drugoj osobi --- impulzivnu (ili reaktivnu) agresivnost --- i onu kojoj je svrha
postizanje nekog cilja, pri čemu je nanošenje štete žrtvi sporedno ---
instrumentalnu (ili proaktivnu) agresivnost \citep{rjecnik,brugman2014identifying}.
\citet{reeve} navodi da društvene norme uglavnom kontroliraju i zabranjuju
otvoreno agresivno ponašanje. Sukladno tome, postoji i 
sklonost k iskrivljavanju odgovora na upitnicima samoprocjene agresivnosti,
upravo zbog njene socijalne nepoželjnosti \citep{banse2015predicting}.

Ali čak i ako ne pokušavaju iskrivljavati svoje odgovore, kad su voljni i
sposobni odgovarati, ispitanici pri rješavanju upitnika samoprocjene 
otkrivaju samo one dijelove koji su dostupni introspekciji, dajući time samo
djelomičnu sliku o svojoj ličnosti \citep{wileyhandzgal}.
Jedan od pristupa koji pokušava zahvatiti
dio ličnosti koji, potencijalno, nije dostupan introspekciji je test implicitnih
asocijacija (IAT).  IAT se temelji na ideji da ljudi stvaraju asocijacije na
temelju svakodnevnih iskustava te da se te asocijacije mogu mjeriti zadacima
brzine reakcije \citep{wileyhandzgal}. \citet{greenwald1995implicit} tvrde da
tragovi prijašnjih iskustava imaju utjecaj na učinak, iako se ljudi tih
iskustava ne mogu sjetiti, odnosno iako ona nisu dostupna introspekciji i mjerama
samoprocjene.  
  
Pri mjerenju testom implicitnih asocijacija nije važno vjeruje li osoba 
u istinitost asocijacija ili ne, niti je li ih  svjesna ili nije \citep{wileyhandzgal}.
Ključna značajka asocijacije je njena snaga, koja se mjeri
računalnim zadatkom kategorizacije. Pri tome, pretpostavka je da će sudionici
brže davati odgovore na snažnije asocijacije nego na slabije
\citep{wileyhandzgal, greenwald1995implicit}. Snaga asocijacija  mjeri se između
dvije ciljne kategorije (npr. \emph{ja} i \emph{drugi}) i dvije bipolarne
kategorije atributa (npr. \emph{opasno} i \emph{bezopasno}).
Sudionici podražaje koji pripadaju naznačenim kategorijama klasificiraju pritiskujući dvije
tipke na tipkovnici. Konačni IAT rezultat je razlika u prosječnom vremenu
reakcije u dva kombinirana zadatka klasifikacije --- na primjer, razlika između
vremena reakcije na
[ja ili opasno / drugi ili bezopasno] i [ja ili bezopasno / drugi ili opasno]
\citep{richetin2008automatic}, pri čemu bi pojedinci koji sebe snažnije asociraju s
\emph{opasnim} nego s \emph{bezopasnim} trebali brže kategorizirati podražaje u
prvom zadatku nego u drugom. Primjer IAT-a za agresivnost prikazan je na slici
\ref{iatapic}.

\begin{figure}
    \centering
    \hspace*{-0.7cm}\begin{subfigure}{0.45\linewidth}
        \centering
        \includegraphics[keepaspectratio, scale = 0.15]{iatscreen}
        \caption{}
        \end{subfigure}
        \hspace*{2em}
    \begin{subfigure}{0.45\linewidth}
        \centering
        \includegraphics[keepaspectratio, scale = 0.15]{iatscreen2}
        \caption{}
        \end{subfigure}
        \vspace*{-0.6cm}
        \caption{\label{iatapic} Primjer zadataka kategorizacije u implicitnom testu asocijacija za
    agresivnost.}
\end{figure}

\citet{banse2015predicting} proveli su nekoliko studija s ciljem validiranja
IAT-a za mjerenje agresivnosti.
Njihova  istraživanja pokazala su da rezultati na IAT-u za
agresivnost koreliraju s jednim objektivnim pokazateljem agresivnog ponašanja
hokejaša (Studija 1), s agresivnim ponašanjem ispitanika u eksperimentalnoj situaciji ($r =$
0.33 odnosno 0.38, za dvije verzije IAT-a; Studija 2) te s procjenama
agresivnosti od strane vršnjaka ($r =$ 0.42; Studija 3).
Ipak, rezultati ne daju potpuno jasnu sliku o povezanosti IAT-a s mjerama
socijalno poželjnog odgovaranja. Osim toga, rezultati Studije 1 nisu pokazali da postoji
statistički značajna razlika između hokejaša i odbojkaša u agresivnosti mjerenoj
IAT-om, iako bi se, s obzirom na prirodu sportova, ta razlika mogla očekivati. 
Na upitnicima samoprocjene, s druge strane, pronađena je statistički značajna
razlika.

\citet{richetin2010predictive} provele su istraživanje u kojem su sudionici
rješavali IAT-e za direktnu i indirektnu agresivnost, nakon čega su trebali
pozvati eksperimentatora. On se ponašao ili bezobrazno (provokacija) ili
neutralno. Nakon te interakcije, drugi je eksperimentator zamolio sudionike da
evaluiraju prvog, uz objašnjenje da moraju odabrati asistente koji će zadržati
svoj posao tijekom sljedećeg semestra. Kao mjera agresivnog ponašanja uzete su evaluacije
prvog eksperimentatora, pri čemu se negativnije evaluacije uzimaju kao indikator
viših razina agresivnosti. Nakon evaluacija, sudionici su ispunili i
eksplicitne mjere direktne i indirektne agresivnosti. Obrada je pokazala da su
IAT-i prediktivni za agresivno ponašanje samo u situaciji provokacije.
Eksplicitne mjere nisu značajno korelirale s evaluacijama eksperimentatora niti
u jednoj situaciji.

Intenzivna istraživanja valjanosti IAT-a za mjerenje agresivnosti svakako su
potrebna, ali dosadašnji rezultati su podosta ohrabrujući. IAT za mjerenje
agresivnosti mogao bi biti posebno značajan zbog snažne sklonosti iskrivljavanju
odgovora na upitnicima samoprocjene ovog konstrukta. 

\subsection{Naše istraživanje}

Kao što je ranije rečeno, želimo proširiti spoznaje vezane uz eksperimentalne
igre i implicitne mjere agresivnosti.
Agresivnost je razmotrena kao potencijalni prediktor odluka u 
igrama  jer se pokazala
prediktivnom za ponašanja u različitim eksperimentalnim situacijama
(kad su u pitanju implicitne mjere, npr. 
\citealp{richetin2010predictive, banse2015predicting}). 
Iako se ta istraživanja nisu bavila nekooperativnim ponašanjima kako bismo ih
mogli definirati u okviru teorije igara, njihovi nalazi potencijalno se mogu
generalizirati i na eksperimentalne igre.

Također,  možemo se osvrnuti na radove koji su se
bavili značajem povjerenja i atribucija u eksperimentalnim igrama.
Unatoč tome što ta istraživanja nisu ispitivala agresivnost, konstrukti koje proučavaju mogli
bi s njom biti povezani.
Primjerice, pokazalo se da očekivanje suradnje od druge strane pozitivno korelira s vlastitom
suradnjom te da je povjerenje važno kad ljudima nedostaju informacije o
namjerama i ponašanju drugih \citep{balliet2013trust, vancoop}. Nadalje,
\citet{kelley1970inference} su pokazali da nekooperativni pojedinci namjere o
suradnji drugih osoba često pogrešno procjenjuju kao nekooperativne.
Možemo povući paralelu s pristranosti hostilne atribucije koju pokazuju agresivni
pojedinci --- dvosmislena ponašanja drugih skloni su interpretirati kao namjerno
hostilna. Kako očekuju da će drugi ljudi biti hostilni, agresivni pojedinci
skloni su tretiranju drugih na agresivan način \citep{larsbuss}. 
Osim toga, pretpostaviti da se agresivnost nalazi u podlozi nekooperativnih
ponašanja u stvarnim situacijama intuitivno djeluje smisleno.

Odlučili smo uključiti i ponašanje suigrača kao prediktor jer neka istraživanja
implicitnih mjera ukazuju na to da one predviđaju ponašanja samo u kontekstu
provokacije (npr. prije spomenut rad \citealp{richetin2010predictive}). Također,
\citet{vancoop} navode da bi agresivnost mogla biti značajan faktor pri
donošenju odluka u eksperimentalnim igrama, ali možda samo pod utjecajem
ponašanja drugih. Za eksplicitne mjere agresivnosti kao osobine
ličnosti uglavnom se pokazuje da predviđaju agresivno ponašanje neovisno o tome
jesu li uvjeti neutralni ili provokativni \citep{bettencourt2006personality}.

\section{Ciljevi i hipoteze}

Cilj ovog istraživanja je proširiti postojeće spoznaje iz bihevioralne teorije
igara te spoznaje vezane uz implicitne i eksplicitne mjere agresivnosti.
Ovim istraživanjem želimo ispitati u kojoj je mjeri moguće predviđati
nekooperativne odgovore u eksperimentalnim igrama na temelju implicitnih i 
eksplicitnih testova agresivnosti. Pitanja koja postavljamo su: 

\begin{enumerate}[label = (\arabic*)]
    \itemsep0em
    \item Je li na temelju implicitne i eksplicitne mjere agresivnosti te
        njihove interakcije moguće predvidjeti odluke o (ne)surađivanju u eksperimentalnim
        igrama?
    \item Ovisi li prediktivna valjanost implicitnih mjera agresivnosti o
        situaciji provokacije?
\end{enumerate}

Iako, prema našim saznanjima, agresivnost nije intenzivno ispitivana kao
potencijalno relevantan faktor u bihevioralnoj teoriji igara, smatramo da
postoje zdravi temelji za pretpostavljanje njene uloge u donošenju odluka u
eksperimentalnim igrama. Te temelje pronalazimo u ranije spomenutim
istraživanjima koja su proučavala konstrukte potencijalno vezane uz agresivnost,
kao što su povjerenje i atribucije. Također, polazimo od istraživanja koja su
proučavala agresivnost, ali ne u kontekstu eksperimentalnih igara. Uzimajući u obzir
gore navedeno, postavljamo sljedeće hipoteze:
\begin{enumerate}[label = {}]
        \itemsep0em
        \item \emph{H1a:} Sudionici koji postižu viši ukupni rezultat na eksplicitnoj
            mjeri agresivnosti imat će veću vjerojatnost donošenja nekooperativnih
            odluka od sudionika koji postižu niži rezultat na istoj mjeri.
        \item \emph{H1b:} Sudionici koji postižu više rezultate na implicitnoj
            mjeri agresivnosti imat će veću vjerojatnost donošenja
            nekooperativnih odluka nego sudionici koji postižu niže rezultate na
            istoj mjeri.
        \item \emph{H1c:} Ako eksplicitna mjera predviđa donošenje
            nekooperativnih odluka u eksperimentalnim igrama, implicitna mjera
            imat će inkrementalnu valjanost u odnosu na nju.
        \item \emph{H1d:} Interakcija implicitne i eksplicitne mjere agresivnosti
            dodatno će objasniti vjerojatnost donošenja nekooperativnih odluka
            eksperimentalnim igrama, povrh pojedinačnih prediktora eksplicitne i
            implicitne agresivnosti: sudionici s višim (ukupnim) rezultatom na
            implicitnoj i eksplicitnoj mjeri češće će donositi nekooperativne
            odluke od onih koji imaju niske rezultate na obje mjere. 
        \item \emph{H2a:} Interakcija rezultata na implicitnoj mjeri agresivnosti
            i ponašanja suigrača u prethodnim igrama bit će prediktivna za
            donošenje nekooperativnih odluka u drugoj i trećoj igri. Sudionici koji postižu visok
            rezultat na IAT-A češće će donositi nekooperativne odluke u drugoj i
            trećoj igri ako je
            suigrač u prethodnim igrama donio nekooperativne odluke, nego ako je
            donio kooperativne odluke. Osobe s niskim rezultatom na implicitnoj
            mjeri agresivnosti
            rjeđe će donositi nekooperativne odluke, neovisno o ponašanju
            suigrača. 
        \item \emph{H2b:} Eksplicitna mjera neće biti u interakciji s ponašanjem
            suigrača u prethodnim igrama.
    \end{enumerate}

\section{Metoda}

\subsection{Sudionici}
U istraživanju je sudjelovalo 109 osoba, a od toga su 55.96\% bile žene.
Sudionici su uglavnom bili studenti (89.91\%). Raspon dobi kretao se od 17 do
44 ($M =$ 21.89, $SD =$ 3.2). Troje sudionika je isključeno iz obrade zbog
tehničkih poteškoća tijekom mjerenja, dodatnih troje je isključeno zbog
netočnih odgovora na kontrolna pitanja u upitnicima, a jedan je isključen zbog
više od 25\% pogrešnih odgovora na IAT-u, što konačan broj ispitanika svodi na
102.

U pozivu na sudjelovanje u istraživanju bilo je navedeno da će, za svakih 30
osoba koje sudjeluju, po jedna osoba s najvišim ukupnim rezultatom u igrama\footnote{Ukupni
 rezultat računat je samo zato da bi se ostavio dojam da se
    sudionici protiv nekoga natječu za nagrade. On nema statističku racionalu
    niti težinu te nije korišten u obradama.}
dobiti poklon bon za jednu drogeriju u vrijednosti od 50 kuna.
Studenti psihologije mogli su
dobiti po jedan eksperimentalni sat (maksimalno 5) za svakog dovedenog
sudionika (ali nisu mogli i sami sudjelovati) --- tim putem je prikupljen
najveći dio uzorka (67.9\%). Značajan dio uzorka
čine studenti fonetike (21.2\%), koji su za sudjelovanje dobili po jedan
eksperimentalni sat. Ostale sudionike čine prijatelji eksperimentatora, koji
nisu bili upućeni u detaljan nacrt i svrhu istraživanja. 

\subsection{Instrumenti}

\emph{Test implicitnih asocijacija za agresivnost --- IAT-A} (Galić, Bubić i
    Parmač Kovačić, 2015)
razvijen je u sklopu projekta
\emph{Implicitna ličnost i radno ponašanje} na Filozofskom fakultetu u Zagrebu.
Ciljne kategorije u ovom testu su \emph{ja} i \emph{drugi}, a kategorije
atributa su \emph{agresivnost} i \emph{miroljubivost}. 

Podražajne riječi koje odražavaju agresivnost uključuju riječi kao što su
\enquote{zlostavljati, tući se, nasilje}, a miroljubivost predstavljaju
riječi kao što su \enquote{mirotvorac,
    blagost, pomiriti se}.
Ciljne kategorije predstavljaju riječi poput \enquote{sebe, moj, ja}, za ciljnu
kategoriju \emph{ja}, i \enquote{njihova, tuđe, oni}, za ciljnu kategoriju
\emph{drugi}. Tablica \ref{iata} prikazuje shemu kategorizacijskih zadataka
za IAT-A. Pouzdanost IAT-A računata kao Cronbachov {\textalpha} iznosi 0.88
(95-postotni interval pouzdanosti kreće se od 0.85 do 0.91).

\begin{table}[h!]
    \centering
    \caption{\label{iata}Shematski prikaz zadataka klasifikacije u IAT-u za
        mjerenje agresivnosti (IAT-A).}
    \hspace*{-0.5cm}\begin{tabular}{l*{4}{c}}
        \toprule
        Blok & Zadatak & N podražaja & Lijeva tipka (E) & Desna tipka (I)\\
        \midrule
        1. & Razvrstavanje ciljnih pojmova 1& 20 & Ja & Drugi\\
        2. & Razvrstavanje atributa & 20 & Agresivno & Miroljubivo\\
        3. & Kompatibilni zadatak 1 & 20 & Ja ili Agresivno & Drugi ili Miroljubivo\\
        4. & Kompatibilni zadatak 2 & 40 & Ja ili Agresivno & Drugi ili Miroljubivo\\
        5. & Razvrstavanje ciljnih pojmova 2 & 20 & Drugi & Ja\\
        6. & Nekompatibilni zadatak 1 & 20 & Drugi ili Agresivno & Ja ili Miroljubivo\\
        7. & Nekompatibilni zadatak 2 & 40 & Drugi ili Agresivno & Ja ili Miroljubivo\\
        \bottomrule
    \end{tabular}
\end{table}

IAT je kreiran pomoću programa \emph{Inquisit 4 Lab} \citep{inq}. 
Blokovi 1, 2 i 5 služe za vježbu, a ostali blokovi su kritični za
računanje rezultata. Sudionicima se na sredini ekrana prikazuju podražajne
riječi, a njihov je zadatak razvrstati ih u kategorije navedene u gornjem
lijevom i gornjem desnom uglu ekrana, koristeći tipke \emph{E} i \emph{I} na
tipkovnici. Nakon pritiskanja tipke, ako je klasifikacija točna,
sudionicima se prikazuje novi podražaj. U slučaju pogrešne klasifikacije, na
sredini ekrana se prikazuje crveni \emph{X}, nakon čega sudionici trebaju
pritisnuti suprotnu tipku kako bi ispravili pogrešku.

Ukupan rezultat na IAT-u računa se kao \emph{D}-indeks ---
razlika između prosječnih vremena reakcije u kompatibilnim i nekompatibilnim
blokovima, podijeljena sa zajedničkom standardnom devijacijom.
Teoretski raspon rezultata kreće se od -2 do +2, pri čemu viši iznosi ukazuju na
izraženije povezivanje pojma o sebi s agresivnosti.

\emph{Buss-Perry upitnik agresivnosti (BPAQ; \citealp{buss1992aggression})}  sastoji se od
29 čestica na koje se odgovara na skali Likertovog tipa s rasponom vrijednosti
od 1 (\enquote{izrazito netipično za mene}) do 7 (\enquote{izrazito
    karakteristično za mene}). U upitnik je dodana jedna kontrolna čestica u
kojoj se od sudionika traži da odaberu određenu vrijednost na skali. Ta čestica
služi za otkrivanje nemarnog ili nasumičnog odgovaranja na pitanja.

Čestice
upitnika grupiraju se u četiri faktora: fizičku agresivnost
(BPAQ-F; npr. \enquote{Ponekad mi se dogodi da ne mogu kontrolirati svoj poriv da udarim
    drugu osobu.}), verbalnu agresivnost (BPAQ-V; npr.
\enquote{Moji prijatelji kažu da sam
    pomalo sklon(a) raspravljanju.}), ljutnju
(BPAQ-A; npr. \enquote{Kada sam frustriran(a),
    ne skrivam svoju iznerviranost.}) i
hostilnost (BPAQ-H; npr. \enquote{Ponekad me izjeda
    ljubomora.}). Ukupan rezultat izražen je kao
prosječna vrijednost odgovorenih pitanja te se, stoga, kreće u rasponu od 1
do 7, pri čemu više vrijednosti ukazuju na snažniju razvijenost konstrukata.
Pouzdanost cijelog testa
računata kao Cronbachov {\textalpha} iznosi 0.89 [0.86--0.92].

\emph{z-Tree, v3.6.7} \citep{fischbacher2007z} je besplatan program za
konstruiranje računalnih eksperimenata iz ekonomije i bihevioralne ekonomije.
Konstruirane su tri eksperimentalne igre: Dilema zatvorenika (DZ), Igra kukavice
(IK) i Igra vođe (IV). Scenariji igara s pripadajućim matricama ishoda nalaze se
u Dodacima. Sudionicima su prezentirani scenariji pojedinih igara te su im
u svakoj igri ponuđene dvije  opcije. Svoje odabire vršili su unosom
brojke u za to namijenjeno polje.

Scenarij Dileme zatvorenika ukratko je opisan u Uvodu te ga ovdje neću
ponavljati. U Igri kukavice, radi se o dva auta koji jure jedan prema drugome.
Sudionici su trebali odabrati hoće li nastaviti voziti ravno i riskirati sudar ili
skrenuti, ali potencijalno biti prozvani kukavicom. Igra vođe je opis situacije
u prometu --- dva auta žele se priključiti na glavnu prometnicu na kojoj se
otvorilo mjesto za samo jedno vozilo. Ako krenu
istovremeno, sudarit će se, ali ako nijedan ne  krene, obojica će ostati stajati
i propustit će priliku za uključivanje. Sudionik je, dakle, trebao odabrati hoće
li se pokušati priključiti ili će dati prednost drugom vozilu.

\emph{Pomoćne i testne matrice.} Fizički primjerci matrica u Tablicama
\ref{pdtab}, \ref{iktab} i \ref{ivtab} (Dodaci) bili su dostupni sudionicima tijekom
istraživanja. Oni su služili kao pomoćno sredstvo za provjeravanje 
mogućih ishoda sudionikovih i \enquote{suigračevih} odluka. Testne matrice bile
su apstraktne matrice (ponuđene odluke su \enquote{odluka A} i \enquote{odluka
    B}; ishodi igara koje imaju strukturu sličnu IK i IV  zamijenjeni su
 slovima, npr. \enquote{K, L, M, T}) jednake strukture kao i prave, a služile su za
objašnjavanje i provjeravanje shvaćanja načina čitanja matrica.

\subsection{Postupak}

Istraživanje je provedeno u malom praktikumu na Odsjeku za psihologiju
Filozofskog fakulteta u Zagrebu.
U pozivu na sudjelovanje u istraživanju sudionicima je rečeno da sudjelovanje 
podrazumijeva igranje nekoliko igara s anonimnim suigračem te
rješavanje nekoliko testova, ali suigrač u stvarnosti nije postojao. 
Suigrač je simuliran jer smatramo da donošenje odluka u igrama protiv čovjeka
bolje odražava procese prisutne u stvarnim životnim situacijama nego donošenje
odluka u igrama protiv računala. Sudionici, dakle, ni u jednom trenu nisu
vidjeli svog suigrača, budući da je on bio samo računalni program.

Ipak, kako bismo pojačali
dojam da suigrač zapravo postoji, na stolu ispred odjeljka
4 do 5 metara udaljenog od onog u kojem su bili sudionici, postavljene su stvari
koje su trebale implicirati da se na tom kraju prostorije nalazi još jedna
osoba. 
Osim toga, u uputi za dolazak do
mjesta mjerenja, sudionici su zamoljeni da dođu u točno vrijeme termina te da
pokušaju ne doći ranije, uz objašnjenje da se to radi zato da ne bi došlo do
slučajnog susreta s njihovim suigračem. Tijekom mjerenja, nakon što je
sudionicima zadan  IAT, eksperimentator je uklonio stvari \enquote{suigrača},  
prošetao do ulaznih vrata prostorije, otvorio ih i zatvorio, kako bi se stvorio
dojam da netko izlazi.

Nakon što su sudionici uvedeni
u svoj odjeljak, dobili su formular za pristanak te im je rečeno da
pokucaju na vrata odjeljka kad završe s čitanjem. 
Prije pokretanja eksperimentalnih igara, sudionicima je dana prilika da postave
pitanja te su s eksperimentatorom prošli kroz dvije testne matrice
kako bi se utvrdilo
da su  naučili ispravno čitati pomoćne matrice. Naposljetku,
sudionicima je rečeno da  u bilo kojem trenu mogu
kucanjem na vrata pozvati eksperimentatora. Također,
rečeno im je da, u slučaju da eksperimentator ne dođe nakon prvog kucanja, malo pričekaju te
pokucaju ponovno jer je moguće da je \enquote{preko puta}, tj. kod njihovog
suigrača.
Prije prve igre, sudionicima je prikazana uputa u kojoj im je rečeno da odaberu
onu odluku koja najbolje odgovara onome što bi napravili u stvarnoj situaciji.
Odgovarajuće upute za IAT i za upitnik samoprocjene dane su prije rješavanja
svakog od tih testova.

Odluke u eksperimentalnim igrama odabirane su unosom odgovarajućeg broja odluke
(1 ili 2) u za to namijenjeno polje. Nakon unosa odgovora, sudionicima je
prikazana poruka u kojoj im je rečeno da pričekaju dok njihov suigrač donosi
odluku. Prikaz je trajao 5 sekundi. Po završetku svake igre, sudionicima su
prikazani odluka njihovog suigrača i ishod igre. 
Nakon posljednje eksperimentalne igre, eksperimentator je  na formular za pristanak zapisao
ukupni rezultat sudionika, kako bi se održao dojam da se
nagrađuju oni s najvišim ukupnim rezultatom. Potom je sudionicima zadan IAT
te na kraju upitnik samoprocjene.
Redoslijed instrumenata uvijek je bio igre--IAT--samoprocjena, zbog mogućnosti
utjecaja rješavanja IAT-a i upitnika na donošenje odluka te mogućnosti utjecaja
upitnika samoprocjene na uradak u IAT-u. Redoslijedi
igara bili su rotirani.

Po završetku prikupljanja podataka, sudionicima je poslan e-mail u kojem je
rečeno da je došlo do manipulacije, objašnjeno je zbog čega je to bilo
potrebno  te su kontaktirani dobitnici poklon
bonova. Odabir dobitnika izvršen je nasumce, budući da je dio sudionika bio u izrazito
povoljnoj situaciji za postizanje visokog rezultata (suigrač koji uvijek
surađuje), a dio u izrazito nepovoljnoj situaciji (suigrač koji nikad ne
surađuje). Nakon završetka obrade, sudionicima je poslan sažetak glavnih nalaza
istraživanja.

\section{Rezultati}

\texttt{R} kod za prikazane obrade, ali i za neke od kojih smo odustali,
dostupan je na \url{https://github.com/vdeni/dip/blob/master/obrada/dipobrada.R}.

\subsection{Deskriptivna statistika i provjera obmane} 

Kako bi se provjerila uspješnost obmane sudionika, njihovi odgovori na upitnik
za provjeru obmane kodirani su od strane dvoje studenata psihologije i jednog
eksperimentatora, koji su
sudionike kategorizirali kao one koji su otkrili obmanu, one koji nisu i one
koji su možda otkrili obmanu.

Za one koji su kategorizirani u \emph{možda} skupinu pretpostavljeno je da nisu
otkrili obmanu. Sudionicima koji nisu odgovorili na upitnik za obmanu
nasumično su dodijeljene oznake \enquote{otkrio} i \enquote{nije otkrio},
u skladu s proporcijom pojedine oznake kod svakog procjenjivača.
Provedeni su hi-kvadrat testovi kako bi se ispitalo postoje li
razlike u učestalosti suradničkih odnosno ne-suradničkih odluka između sudionika
za koje je procijenjeno da su otkrili obmanu i onih koji nisu. 
Kriteriji su bile odluke donesene u svakoj igri, te odluke
donesene u prvoj, drugoj i trećoj igri po redu, zanemarujući pritom o kojoj se
igri radi (npr. ako je kriterij odluka donesena u prvoj igri, a prva igra nekog
sudionika bila je Dilema zatvorenika, njegov rezultat u kriteriju je odluka
donesena u toj igri).

Prema procjeni, obmanu je otkrilo 18.2\% sudionika. 
Ni u jednom kriteriju ne postoje statistički značajne
razlike između onih za koje je procijenjeno da su otkrili obmanu i onih za koje
je procijenjeno da nisu.
Također, s obzirom na to da su rezultati nekih istraživanja
\citep{ben2008economic} pokazali da je ponašanje sudionika u nekim
eksperimentalnim igrama isto, neovisno o tome jesu li situacije stvarne ili
hipotetske, u obradi su zadržani svi sudionici.

Tablica \ref{deskr krit} prikazuje postotak nekooperativnih odluka u
pojedinom kriteriju, a Tablica \ref{deskr pred} deskriptivnu statistiku za mjerene
prediktore.
Postoci nekooperativnih odluka u igrama su 22.54\% za Dilemu zatvorenika, 13.73\% za Igru
kukavice  i 39.22\% za Igru vođe. 
Aritmetička sredina na IAT-A iznosi -0.42, što ukazuje na
to da sudionici u prosjeku pojam o sebi više asociraju s miroljubivosti
nego s agresivnosti. 
Aritmetička sredina BPAQ\textsubscript{ukupno} iznosi 3.14, što je nešto niže od
prosječne vrijednosti skale (3.5).
Distribucije rezultata na obje su mjere pozitivno
asimetrične, što znači da je agresivnih sudionika bilo manje od neagresivnih.

\begin{table}
    \caption{Postotak nekooperativnih odluka u igrama korištenima u istraživanju. (N =
        102)\label{deskr krit}}
    \centering
    \begin{tabular}{l*{1}{r}}
        \toprule
        kriterij & postotak\\
        \midrule
        odluka u Dilemi zatvorenika & 22.54\% \\
        odluka u Igri kukavice & 13.73\% \\
        odluka u Igri vođe & 39.22\% \\
        odluka u prvoj igri & 20.59\% \\
        odluka u drugoj igri & 25.49\% \\
        odluka u trećoj igri & 29.41\% \\
        \bottomrule
    \end{tabular}
\end{table}

\begin{table}[h]
    \caption{Deskriptivna statistika za prediktore korištene u 
        istraživanju. (N = 102)\label{deskr pred}}
    \centering
    \begin{tabular}{l*{7}{r}}
        \toprule
        mjera & $M$ & $SD$ & $min$ & $max$ & zakrivljenost & spljoštenost\\
        \midrule
        IAT-A & -0.42 & 0.33 & -1.03 & 0.7 & 0.68 & 0.52\\
        BPAQ\textsubscript{ukupno} & 3.14 & 0.82 & 1.62 & 5.41 & 0.50 & -0.01\\
        \bottomrule
    \end{tabular}
\end{table}

Korelacija IAT-A s ukupnim rezultatom na
BPAQ je neznačajna ($r =$ -0.06, $p =$ 0.50). IAT-A nije u značajnoj korelaciji
niti s pojedinim faktorima BPAQ.
Bivarijatne korelacije odluka u igrama računate
kao fi-koeficijent nisu značajne.

\subsection{Agresivnost kao prediktor odluka u eksperimentalnim igrama}

U svim analizama, kako bismo kompenzirali za potencijalno nisku statističku
snagu, odlučili smo pratiti savjet koji daje 
\citet{aguinis1995statistical} te traženu razinu značajnosti za interakcije
postaviti na 0.1. Također, prateći savjete Jaccarda
\citeyearpar{jaccard2001interaction} te Tabachnick i Fidell
\citeyearpar{tabachnick2012multi}, kontinuirane varijable unesene u regresijske
modele centrirane su na nulu. 

Kako bismo provjerili Hipotezu 1a, izračunali smo bivarijatne korelacije BPAQ\textsubscript{ukupno}
s odlukama u pojedinim eksperimentalnim igrama. Logističke regresijske analize
pokazale su da BPAQ\textsubscript{ukupno} nije prediktivan za donošenje
nekooperativnih odluka u
Igri kukavice ($B =$ 0.407, $p =$ 0.23) i Dilemi zatvorenika ($B =$ -0.100, $p
=$ 0.74). Ukupan rezultat na BPAQ pokazao se prediktivnim za donošenje
nekooperativnih odluka u Igri vođe ($B =$ -0.651, $p =$ 0.02), pri čemu
vjerojatnost donošenja nekooperativne odluke pada s porastom
BPAQ\textsubscript{ukupno}. Detalji regresija prikazani su u Tablici
\ref{bivargameks}. 
Zaključujemo da Hipoteza 1a nije potvrđena. BPAQ\textsubscript{ukupno} pokazao se prediktivnim
samo za donošenje odluka u Igri vođe, ali ta povezanost je suprotnog smjera od
očekivanog.

\begin{table}
    \begin{center}
        \caption{\label{bivargameks} Logističke regresijske analize s
            BPAQ\textsubscript{ukupno} kao prediktorom nekooperativnih odluka u
            eksperimentalnim igrama. (N = 102)}
        \hspace*{-0.7cm}\begin{tabular}{lrrrr}
        \toprule
        Kriterij & $B$ & $SE$ & $z$ & OR (95\% CI)\\
        \midrule
        IK & 0.407 & 0.341 & 1.193 &\\
        DZ & -0.099 & 0.294 & -0.336 &\\
        IV* & -0.651 & 0.279 & -2.333 & 0.521 (0.293, 0.881)\\
        \bottomrule
        \multicolumn{5}{l}{
            \parbox{9cm}{\scriptsize \vspace{3pt} 
                * p < 0.05\\
                IK = Igra kukavice; DZ = Dilema zatvorenika; IV = Igra vođe
        }}
    \end{tabular}
\end{center}
\end{table}

Za testiranje Hipoteze 1b, testirali smo modele ekvivalentne prethodno
navedenima, ali s IAT-A kao prediktorom. IAT-A nije se pokazao prediktivnim za
donošenje nekooperativnih odluka u niti jednoj igri (za IK: $B =$ -0.271, $p =$
0.76; za DZ: $B =$ 0.114, $p =$ 0.87; za IV: $B =$ 0.283, $p =$ 0.643). Detalji analiza nalaze se u
Tablici \ref{bivargamimp}. Hipoteza 1b također nije potvrđena. Rezultat na IAT-A
nije se pokazao prediktivnim za donošenje nekooperativnih odluka u korištenim
eksperimentalnim igrama.

\begin{table}
    \begin{center}
        \caption{\label{bivargamimp} Logističke regresijske analize s
            IAT-A kao prediktorom nekooperativnih odluka u
            eksperimentalnim igrama. (N = 102)}
        \hspace*{-0.7cm}\begin{tabular}{lrrr}
        \toprule
        Kriterij & $B$ & $SE$ & $z$ \\
        \midrule
        IK & -0.271 & 0.892 & -0.304 \\
        DZ & 0.114 & 0.712 & 0.160 \\
        IV & 0.283 & 0.612 & 0.463 \\
        \bottomrule
        \multicolumn{4}{l}{
            \parbox{5cm}{\scriptsize \vspace{3pt} 
                IK = Igra kukavice; DZ = Dilema zatvorenika; IV = Igra vođe
        }}
    \end{tabular}
\end{center}
\end{table}

Hipoteza 1c nije potvrđena samim time što se zasebne mjere nisu pokazale
prediktivnima za donošenje odluka u eksperimentalnim igrama (osim
BPAQ\textsubscript{ukupno} za Igru vođe). Za testiranje interakcijske Hipoteze
1d, provedene su  višestruke regresijske analize. U jednom koraku unesene su
obje mjere te njihova interakcija. Nijedna interakcija nije se pokazala
prediktivnom za donošenje nekooperativnih odluka u eksperimentalnim igrama (za
IK: $B =$ 0.052, $p =$ 0.968; za DZ: $B =$ -1.244, $p =$ 0.25; za IV: $B =$
0.530, $p =$ 0.58). Detalji analiza prikazani su u tablici \ref{intgam}.

\begin{table}
    \begin{center}
        \caption{\label{intgam} Sažetak modela višestruke logističke regresije s
            IAT-A i BPAQ\textsubscript{ukupno} kao prediktorima donošenja
            nekooperativnih odluka u eksperimentalnim igrama. (N = 102)}
        \hspace*{-0.5cm}\begin{tabular}{llrrrr}
        \toprule
        Kriterij & Prediktor & $B$ & $SE$ & $z$ &\\
        \midrule
        IK & konstanta* & -1.879 & 0.298 & -6.305 &
        \multirow{4}{*}{\shortstack[l]{$R^2_N =$ 0.03 \\ $\chi^2(3) =$ 1.458
            \\ $p =$ 0.69}}\\
        &BPAQ\textsubscript{ukupno} & 0.404 & 0.347 & 1.163 \\
        &IAT-A & -0.212 & 0.938 & -0.226 &\\
        &BPAQ\textsubscript{ukupno} x IAT-A & 0.052 & 1.279 & 0.041\\
        &&&&&\\ 
        DZ & konstanta* & -1.276 & 0.245 & -5.213 &
        \multirow{4}{*}{\shortstack[l]{$R^2_N =$ 0.02 \\ $\chi^2(3) =$ 1.494
            \\ $p =$ 0.68}}\\
        &BPAQ\textsubscript{ukupno} & -0.134 & 0.300 & -0.445 \\
        &IAT-A & 0.205 & 0.740 & 0.277 &\\
        &BPAQ\textsubscript{ukupno} x IAT-A & -1.244 & 1.087 & -1.145\\
        &&&&&\\ 
        IV & konstanta* & -0.469 & 0.211 & -2.218 &
        \multirow{4}{*}{\shortstack[l]{$R^2_N =$ 0.08 \\ $\chi^2(3) =$ 6.467
            \\ $p =$ 0.09}}\\
        &BPAQ\textsubscript{ukupno}* & -0.642 & 0.281 & -2.286 \\
        &IAT-A & 0.197 & 0.635 & 0.311 &\\
        &BPAQ\textsubscript{ukupno} x IAT-A & 0.530 & 0.953 & 0.556\\
        \bottomrule
        \multicolumn{5}{l}{
            \parbox{7cm}{\scriptsize \vspace{3pt} 
                * p < 0.05\\
                IK = Igra kukavice; DZ = Dilema zatvorenika; IV = Igra vođe
        }}
    \end{tabular}
\end{center}
\end{table}

\subsection{Predviđanje reakcije na suigračevo ponašanje}

Za testiranje Hipoteza 2a i 2b provedene su hijerarhijske, višestruke logističke
regresijske analize. U prvom koraku uneseni su pojedinačni prediktori: IAT-A,
BPAQ\textsubscript{ukupno} i odluka suigrača (\texttt{botdec}). U drugom koraku
unesene su sve moguće dvostruke interakcije tih varijabli. Kriteriji u ovim
modelima su odluke u drugoj i trećoj igri po redu, neovisno o tome o kojoj se
igri radi.

Dvostruka interakcija IAT-A i ponašanja suigrača nije se pokazala prediktivnom
($B =$ -0.017, $p =$ 0.99)
za donošenje nekooperativnih odluka u drugoj igri, što nije u skladu s Hipotezom
2a. Prediktivnom se nije pokazala ni dvostruka interakcija BPAQ\textsubscript{ukupno}
i ponašanja suigrača, što je u skladu s Hipotezom 2b ($B =$ -0.412, $p $ = 0.38).
Isti slučaj opažamo pri predviđanju odluke u trećoj igri po redu (za IAT-A i
odluku suigrača, $B =$ 0.342, $p =$ 0.82; za BPAQ\textsubscript{ukupno} i odluku
suigrača, $B =$ -0.299, $p =$ 0.63). Sažeci modela prikazani su u Tablicama
\ref{glmtotgam22} i \ref{glmtotgam33}.

\begin{table}
    \begin{center}
        \caption{\label{glmtotgam22} Model logističke regresije s dvostrukim interakcijama IAT-A,
            BPAQ\textsubscript{ukupno} i ponašanja suigrača kao prediktorima
            nekooperativne odluke u drugoj igri po redu. Za puni
            model: $\chi^2 =$
            3.282, $df =$ 6, $p =$ 0.12, $N =$ 102.}
        \hspace*{-0.5cm}\begin{tabular}{llrrr}
        \toprule
        Model & Prediktor & $B$ & $SE$ & $z$\\
        \midrule
        I & konstanta* & -1.419 & 0.359 & -3.954 \\
        \multirow{3}{*}{$R^2_N =$ 0.03}
        &BPAQ\textsubscript{ukupno} & -0.138 & 0.286 & -0.483 \\
        &IAT-A & 0.353 & 0.699 & 0.505 \\
        &botdec & 0.627 & 0.472 & 1.328 \\
        &&&&\\ 
        II & konstanta* & -1.407 & 0.364 & -3.862 \\
        \multirow{6}{*}{$R^2_N =$ 0.04}
        &BPAQ\textsubscript{ukupno} & 0.099 & 0.459 & 0.217 \\
        &IAT-A & 0.421 & 1.004 & 0.420 \\
        &botdec & 0.601 & 0.478 & 1.257  \\
        &IAT-A x botdec & -0.191 & 1.439 & -0.133 \\
        &BPAQ\textsubscript{ukupno} x botdec & -0.407 & 0.608 & -0.670 \\
        &BPAQ\textsubscript{ukupno} x IAT-A & -0.209 & 1.061 & -0.197 \\
        \bottomrule
        \multicolumn{5}{l}{
            \parbox{3cm}{\scriptsize \vspace{3pt} 
                botdec = odluka suigrača
        }}
    \end{tabular}
\end{center}
\end{table}

\begin{table}
    \begin{center}
        \caption{\label{glmtotgam33} Model logističke regresije s dvostrukim
            interakcijama IAT-A,
            BPAQ\textsubscript{ukupno} i ponašanja suigrača kao prediktorima
            nekooperativne odluke u trećoj igri po redu. Za puni
            model: $\chi^2 =$
            7.664, $df =$ 6, $p =$ 0.26, $N =$ 102.}
        \hspace*{-0.5cm}\begin{tabular}{llrrrr}
        \toprule
        Model & Prediktor & $B$ & $SE$ & $z$ & OR (95\% CI)\\
        \midrule
        I & konstanta* & -1.424 & 0.340 & -3.962 &\\
        \multirow{3}{*}{$R^2_N =$ 0.07}
        &BPAQ\textsubscript{ukupno} & -0.295 & 0.282 & -1.048 &\\
        &IAT-A & 0.111 & 0.690 & 0.161 &\\
        &botdec* & 0.967 & 0.463 & 2.089 & 2.631 (1.083, 6.742)\\
        &&&&&\\ 
        II & konstanta* & -1.409 & 0.368 & -3.676 & \\
        \multirow{6}{*}{$R^2_N =$ 0.10}
        &BPAQ\textsubscript{ukupno} & -0.170 & 0.473 & -0.360 & \\
        &IAT-A & 0.010 & 1.062 & 0.010 &\\
        &botdec$^\ddagger$ & 0.884 & 0.475 & 1.862  & 2.421 (0.972, 6.367)\\
        &BPAQ\textsubscript{ukupno} x IAT-A & -1.609 & 1.254 & -1.430 & \\
        &BPAQ\textsubscript{ukupno} x botdec & -0.299 & 0.620 & -0.482 & \\
        &botdec x IAT-A & 0.342 & 1.497 & 0.228 & \\
        \bottomrule
        \multicolumn{5}{l}{
            \parbox{3cm}{\scriptsize \vspace{3pt} 
                * p < 0.05\\
                $\ddagger$  p < 0.07\\
                botdec = odluka suigrača
        }}
    \end{tabular}
\end{center}
\end{table}

\subsection{Dodatne analize}

Odlučili smo provesti dva seta analiza koja nisu bila predviđena u hipotezama,
ali za koje smatramo da mogu biti teoretski opravdane. Prvo, odlučili smo
ispitati je li položaj igre u slijedu moderator povezanosti mjera agresivnosti i
donošenja nekooperativnih odluka u eksperimentalnim igrama. Mislimo da je moguće
da položaj igre u slijedu utječe na ponašanje sudionika. Primjerice, sudionici 
pri donošenju odluke u prvoj igri ne znaju kako će se njihov suigrač ponašati.
Pri donošenju odluke u drugoj igri, možda već imaju jasniju sliku o
\enquote{karakteru} svog suigrača. U posljednjoj igri, moguće je da su sudionici
spremniji riskirati ili donositi nekooperativne odluke jer znaju da se radi o
zadnjoj igri. Za primjer, možemo pogledati Igru kukavice --- kad je IK bila prva
po redu, tek 3.1\% sudionika donijelo je nekooperativnu odluku. Kad je bila
druga ili treća, 17.9\% odnosno 19.0\% sudionika donijelo je nekooperativnu
odluku. Stoga, smatramo da je valjano razmotriti i položaj igre kao
potencijalni faktor. Sve tablice i slike vezane uz dodatne analize nalaze se u
Dodacima. 

Drugo, odlučili smo provjeriti je li trostruka interakcija
IAT-A, BPAQ\textsubscript{ukupno} i ponašanja suigrača prediktivna za donošenje
odluka u drugoj i trećoj igri po redu. Smatramo da je moguće da ponašanje suigrača
utječe na međuodnos implicitne i eksplicitne agresivnosti te da dolazi do
diferencijalne prediktivnosti ovisno o tome koje je odluke suigrač prethodno
donosio. 

\paragraph{Interakcija IAT-A, BPAQ\textsubscript{ukupno} i položaja igre} 
Provedena je hijerarhijska, višestruka logistička regresijska analiza. U prvom
koraku uneseni su IAT-A, BPAQ\textsubscript{ukupno} i varijabla koja označava
položaj pojedine igre u slijedu igara. U drugom koraku unesene su sve dvostruke
interakcije, a u trećem trostruka interakcija.
Jedini model koji se pokazao prediktivnim je onaj koji predviđa donošenje
nekooperativnih odluka u Dilemi zatvorenika 
($B =$ -3.815, $p =$ 0.023). Sažetak modela prikazan je u Tablici
\ref{glmtotposP}, a grafički prikaz na Slici \ref{glmtotposPplot}. Preostala
dva modela nisu prikazana zbog  uštede prostora.
Ako je Dilema zatvorenika bila prva po redu, vjerojatnost donošenja nekooperativne odluke
najviša je za one s visokim rezultatima na IAT-A i BPAQ\textsubscript{ukupno}
te za one nisko na obje mjere. Za drugo i treće mjesto u slijedu,
najviša vjerojatnost donošenja nekooperativnih odluka predviđena je za pojedince
čiji su rezultati na IAT-A i BPAQ\textsubscript{ukupno} suprotnog
smjera. Pritom je vjerojatnost u sva tri slučaja uvijek viša za one s višim
rezultatom na IAT-A nego za one s nižim rezultatom, a predviđene vjerojatnosti
najviše su kad je Dilema zatvorenika posljednja po redu.

S obzirom na ponešto neočekivan obrazac povezanosti (više o tome u Raspravi),
proveli smo dodatne analize kako bismo pokušali otkriti moguće
statističke nedostatke. U skladu s prijedlogom koji daju \citet{fieldr} te
\citet{tabachnick2012multi}, provjerene su
teoretske frekvencije u kontingencijskoj tablici s odlukom u Dilemi zatvorenika i 
položajem igre u slijedu kao varijablama. Sve teoretske frekvencije veće su od
5, što zadovoljava uvjete koje autori postavljaju.
Testirana je linearnost povezanosti kontinuiranih prediktora s logitom
kriterija te je i ovaj uvjet zadovoljen \citep{fieldr, jaccard2001interaction}.
Kako bi se provjerilo postojanje multikolinearnosti, provjerene su korelacije
između IAT-A, BPAQ\textsubscript{ukupno} i položaja igre te su se sve korelacije
pokazale neznačajnima. Provjereni su iznosi faktora inflacije varijance (VIF) za
prediktore u osnovnom modelu (model I u Tablici \ref{glmtotposP}) --- 
najviša vrijednost iznosi 1.01 (za
BPAQ\textsubscript{ukupno}), što je daleko od graničnog iznosa od oko 10 te
vrlo blizu minimumu funkcije, koji iznosi 1
\citep{james2013introduction, fieldr, salkind2007encyclopedia}. 

Jedan od indikatora multikolinearnosti su i velike  promjene u procijenjenim
koeficijentima kad se neke jedinice analize izbrišu
\citep{salkind2007encyclopedia}. Kako bismo to provjerili, ponovili smo istu
regresijsku analizu 1000 puta, svaki put iz analize nasumično isključujući 10
sudionika. Deskriptivni podaci za provedene analize nalaze se u Tablici
\ref{deskr p robust}. Kao što možemo vidjeti, prosječna vrijednost procijenjenog
koeficijenta trostruke interakcije iznosi -3.88, a njegova standarda devijacija
0.76. Osim toga, u 83.5\% replikacija interakcija je postigla razinu
značajnosti od 0.06 ili niže. Smatramo da ova analiza ukazuje na to da su
procjene koeficijenata razmjerno otporne na promjene u uzorku, a što je argument
protiv postojanja multikolinearnosti.
Ipak, još jedan moguć
indikator postojanja multikolinearnosti je i predznak koeficijenta suprotan od teoretski
očekivanog \citep{salkind2007encyclopedia}, što je svakako slučaj za ovu trostruku interakciju.

\paragraph{Interakcija IAT-A, BPAQ\textsubscript{ukupno} i odluke suigrača} 
Model se nije pokazao prediktivnim za donošenje nekooperativnih odluka u drugoj
igri. Sažetak modela nije prikazan zbog uštede prostora. 
Za odluku u trećoj igri, značajnom se pokazala trostruka interakcija
BPAQ\textsubscript{ukupno}, IAT-A i odluke suigrača u ranijim igrama ($B =$
5.112, $p =$ 0.068). Sažetak modela nalazi se u Tablici \ref{glmtotgam3}. Kao
što je vidljivo na Slici \ref{glmtotgam3plot}, u slučaju prethodne suradnje od
strane suigrača, najviša vjerojatnost za donošenje nekooperativnih odluka
predviđena je za pojedince čiji su IAT-A i BPAQ\textsubscript{ukupno} suprotnih
predznaka --- za one visoko na IAT-A i nisko na BPAQ\textsubscript{ukupno} te
obrnuto. U slučaju prethodne nesuradnje, razlike nisu toliko snažno izražene.

Kao i za model prikazan u Tablici \ref{glmtotposP}, provedene su dodatne analize
kako bi se testirala valjanost modela. I ovdje se pokazalo da su ranije navedeni
uvjeti za računanje logističke regresije zadovoljeni. Također, pokazalo se da ne
postoje značajne bivarijatne korelacije među prediktorima. Najviši VIF u
osnovnom modelu iznosi 1.03, za ponašanje suigrača u prethodnim igrama.

Ponovili smo i replikaciju analize na poduzorcima. Deskriptivna statistika za
1000 ponavljanja regresijske analize prikazana je u Tablici \ref{deskr gam3 robust}. 
Možemo primijetiti veća odstupanja
između koeficijenata dobivenih na punom uzorku i prosječnih vrijednosti nastalih
replikacijama. Osim toga, trostruka interakcija  je statističku
značajnost dosegla tek u 38\% replikacija, čak i uz smanjenu strogost. Uz
dodatno ublažavanje kriterija značajnosti na 0.1, i dalje je samo 59.4\%
repliciranih interakcija bilo značajno. 

\section{Rasprava}

\subsection{Deskriptivna statistika i odnosi među prediktorima}

Pouzdanost IAT-A je zadovoljavajuća i u skladu s prijašnjim nalazima 
(0.91 i 0.89 u \citealp{richetin2010predictive}; 0.80 u
\citealp{banse2015predicting}).
IAT-A ne korelira s ukupnim rezultatom na BPAQ. 
Neznačajna korelacija IAT-A s pretpostavljenom odgovarajućom eksplicitnom mjerom
u skladu je s prijašnjim nalazima --- npr. \citet{banse2015predicting}
navode da je korelacija dvije verzije IAT-A i eksplicitne mjere agresivnosti na
jednom uzorku bila 0.36 odnosno 0.48, dok je korelacija istih mjera u drugom
uzorku bila neznačajna. \citet{richetin2010predictive} također nisu
pronašle značajne korelacije IAT-A i eksplicitnih mjera.
Općenito, niske korelacije  implicitnih i eksplicitnih mjera uobičajene su 
\citep{uhlmann2012getting, mcclelland1989self}.

Što se tiče učestalosti nekooperativnih odluka, nismo naišli na podatke za
Igru kukavice i Igru vođe, ali za Dilemu zatvorenika se može naći 
da postotak nesuradničkih odluka u ponovljenoj Dilemi s poznatim
trajanjem iznosi 57\% (meta-analiza Jonesa, \citeyear{jones2008smarter}), dok se za
\enquote{jednokratne}
(eng. \emph{one-shot}) Dileme zatvorenika može pronaći podatak o 67\%
nekooperativnih odluka \citep{shafir1992thinking}. Moguće je da je postotak
suradničkih odluka u ovom istraživanju pomaknut naviše zbog upute sudionicima da
donesu odluku koju bi donijeli u stvarnoj situaciji, ali i zbog uzorka koji je
većinom bio sastavljen od studenata.

\subsection{Agresivnost kao prediktor odluka u eksperimentalnim igrama}

Očekivanja iz Hipoteze 1 nisu potvrđena. BPAQ\textsubscript{ukupno} 
nije se pokazao prediktivnima za donošenje nekooperativnih odluka u Dilemi
zatvorenika i Igri kukavice (H1a). S druge strane, dok se pokazao
prediktivnim za donošenje nekooperativnih odluka u Igri vođe, korelacija je
negativna, a ne pozitivna kao što je očekivano --- sudionici koji postižu niži
rezultat na BPAQ\textsubscript{ukupno} vjerojatnije će donijeti
nekooperativnu odluku od onih koji postižu viši rezultat.
Razlog za takav obrazac povezanosti nije očit. 

Sama implicitna mjera agresivnosti nije se pokazala prediktivnom za donošenje
nekooperativnih odluka, što također nije u skladu s očekivanjima (H1b).
Implicitna mjera nije pokazala inkrementalnu valjanost u objašnjavanju
kriterija, što je suprotno našim očekivanjima (H1c).
Osim toga, ni dvostruka interakcija IAT-A i BPAQ\textsubscript{ukupno} 
nije se pokazala prediktivnom za donošenje nekooperativnih odluka u
eksperimentalnim igrama (H1d).

U dodatnim analizama testirali smo i modele u 
koje smo uvrstili položaj igre u slijedu kao prediktor.
Za Igru vođe i Kukavice, uvođenje te varijable nije se pokazalo relevantnim. 
Kod Dileme zatvorenika, pak, pokazalo se da je položaj igre u
interakciji s IAT-A i BPAQ\textsubscript{ukupno}. U situaciji kad je
Dilema zatvorenika bila prva po redu, najviša vjerojatnost donošenja
nekooperativne odluke predviđena je za pojedince visoko na IAT-A i
BPAQ\textsubscript{ukupno}, te za one nisko na obje mjere. Prvi slučaj
je u skladu s očekivanjima --- IAT-A i BPAQ\textsubscript{ukupno}
djeluju sinergistički na povećanje vjerojatnosti nekooperativne odluke. Drugi
slučaj nije očekivan, no valja uočiti da najviša predviđena vjerojatnost ovdje ne
prelazi 40\% (u prvom slučaju prelazi 60\%) te da je moguće da razlike u
vjerojatnosti donošenja nekooperativnih odluka između
različitih razina BPAQ\textsubscript{ukupno} na ovom dijelu spektra
IAT-A nisu statistički značajne.

Od druge pozicije nadalje, obrazac se  mijenja te je najviša vjerojatnost
predviđena za pojedince čiji su rezultati na IAT-A i BPAQ\textsubscript{ukupno}
suprotnih smjerova. U obje situacije, predviđena vjerojatnost viša je
za one s višim rezultatom na IAT-A. Osim toga, kad je Dilema zatvorenika
posljednja po redu, predviđene vjerojatnosti su više nego na ranijim pozicijama.

Za pokušaj objašnjavanja ovih obrazaca, referirat ćemo se na 
Opći model prototipova ličnosti (po potrebi, dalje u tekstu \emph{Model} s
velikim \emph{M}; \citealp{bing2007integrating}), koji se bavi 
mogućim obrascima povezanosti implicitnih i eksplicitnih komponenata osobina
ličnosti.
Prema Bingu i sur. \citeyearpar{bing2007integrating}, možemo razlikovati četiri osnovna profila
međuodnosa implicitnih i eksplicitnih mjera iste osobine. Obrasci jednako
razvijenih
implicitnih i eksplicitnih komponenata (npr. visoka implicitna i
visoka eksplicitna agresivnost) nazivaju se kongruentna prisutnost odnosno
kongruentna odsutnost (ako su obje komponente slabo razvijene). 
Profil visoke razine implicitne, a niske
razine eksplicitne komponente autori nazivaju nekongruentnim poricanjem. Obrnuti profil
nazivaju nekongruentnim prenaglašavanjem (\emph{eng.} overclaiming). 

Osobe s profilom kongruentne prisutnosti pokazivat
će ponašanja koja su jasan indikator pojedine osobine ličnosti, a u slučaju
socijalno nepoželjnih motiva, razvit će kognicije kojima opravdavaju takva
ponašanja. Kod osoba s profilom nekongruentnog poricanja, s druge strane, motiv
će se manifestirati na indirektne načine. 

Ovdje valja
napomenuti da u ovoj igri ne možemo razlikovati indirektno i direktno
iskazivanje agresije --- netko tko želi indirektno iskazati agresivnost, to može
učiniti samo donošenjem nekooperativne odluke, isto kao i netko tko agresivnost želi
iskazati direktno.
Osim toga, Dilema zatvorenika nije situacija u
kojoj se može jasno odrediti koji su motivi u podlozi donošenja odluka \citep{colgt}.
Odluke se ne mogu opisati kao isključivo kompetitivne ili kooperativne, kao ni
kao isključivo agresivne ili miroljubive. Stoga, varijacije u agresivnosti
vrlo vjerojatno nisu jedini faktor relevantan za objašnjavanje varijacija u
ponašanju sudionika. 
Imajući to na umu, možemo
zaključiti da su predviđanja našeg logističkog modela u skladu s pretpostavkama 
Modela Binga i sur. \citeyearpar{bing2007integrating}
za situaciju u kojoj je Dilema zatvorenika prva po redu. 

U situaciji u kojoj je Dilema zatvorenika druga po redu, najviša vjerojatnost
donošenja nekooperativnih odluka predviđena je za pojedince s profilom
nekongruentnog poricanja. Kao što je ranije navedeno, pretpostavka Modela je da
će ti pojedinci svoju agresivnosti iskazivati na indirektne načine. 
Osim toga, vjerojatno je da pri prijelazu s prve na drugu igru dolazi do
promjene vrste agresivnosti koja se iskazuje. Naime, budući da 
sudionici tijekom prve igre nisu bili upoznati s ponašanjem, odnosno
\enquote{karakterom} svog suigrača, nekooperativno ponašanje (u ovom kontekstu)
vjerojatno odražava proaktivnu agresiju. U drugoj i trećoj igri, pak,
vjerojatnije je da ono reflektira reaktivnu agresiju, budući da je suigrač dotad
već demonstrirao svoje ponašanje. Stoga, takvi nalazi mogli bi biti u
skladu s onima \citet{richetin2010predictive}, koje su pokazale da IAT za
agresivnost predviđa agresivno ponašanje samo u kontekstu provokacije, odnosno,
reaktivnu agresiju.
Znatan porast vjerojatnosti donošenja
nekooperativne odluke kad je Dilema zatvorenika treća po redu ne čudi, budući da
se donošenjem nekooperativne odluke izbjegava najgori mogući ishod za pojedinca,
što ima posebnu težinu u situaciji u kojoj je suigrač ranije pokazao da
nema namjeru surađivati. Treba istaknuti da je efekt ponašanja suigrača, koje
je važan prediktor suradnje \citep{balliet2013trust}, nemoguće razlučiti iz
prikazane interakcije, no da je on vrlo vjerojatno prisutan.

Uz ove aspekte interakcije koji su u skladu s Modelom, valja ukazati i na one
koji to potencijalno nisu --- \emph{potencijalno}, jer Model u obzir ne uzima
utjecaj situacijskih faktora, što ponašanje suigrača sigurno jest. Osim toga, ne bavi
se odnosom profila ličnosti i ponašanja pojedinca u ponovljenim strateškim
interakcijama, kakve su eksperimentalne igre korištene u ovom istraživanju.
Primjerice, nije jasno zbog čega bi vjerojatnost donošenja nekooperativnih
odluka za osobe s profilom kongruentne prisutnosti bila niska kad je Dilema
zatvorenika druga ili treća po redu. Za osobe s profilom nekongruentnog
prenaglašavanja teško je reći je li predviđanje našeg modela u skladu s
pretpostavkama Modela Binga i sur. \citeyearpar{bing2007integrating}.
S jedne strane, \citet{bing2007integrating} tvrde da će takvi pojedinci, u
slučaju socijalno nepoželjnih motiva, biti pretjerano samokritični,  što
implicira da se neće lako upuštati u takva ponašanja. S druge strane, kad se
radi o socijalno poželjnim motivima, upuštat će se u takva ponašanja, ali to neće
činiti spontano, nego kontrolirano i bez intrinzičnog zadovoljstva. 
Ono što otežava jasno donošenje zaključka o ovom profilu su (a) činjenica da nam
nedostaju podaci o viđenju sebe, pa ne možemo zaključivati o samokritičnosti i
(b) to što odluke u Dilemi zatvorenika nemaju očitu agresivnu komponentu, pa je
moguće da to  \enquote{ublažava} sklonost samokritičnosti i dovodi do
donošenja nekooperativnih odluka.

Kao što je rečeno u Rezultatima, obrazac povezanosti ove trostruke interakcije s
donošenjem nekooperativnih odluka ponešto je neočekivan i nije ga lako
interpretirati u potpunosti u skladu s dosadašnjim spoznajama. 
Ipak, ispitivanja provedena s ciljem otklanjanja mogućnosti utjecaja
multikolinearnosti ukazuju na to da ona ne bi trebala biti problem. 
Također, čini se da se ne radi niti o jednom od slučajeva supresor efekata 
onako kako je definirano kod Salkinda \citeyearpar{salkind2007encyclopedia}.  
Replikacija modela na nasumičnim poduzorcima jednake veličine ukazuje na
otpornost modela na varijacije u strukturi sudionika, što ide u prilog
valjanosti modela. S obzirom na navedeno, moguće je da su neobični obrasci
korelacije produkt same prirode igre, njene valjanosti kao kriterija
agresivnosti te nemogućnosti adekvatnog zahvaćanja ponašanja suigrača u
testiranom modelu.

\subsection{Predviđanje reakcije na ponašanje suigrača} 

U skladu s prijašnjim istraživanjima,
pretpostavili smo da će postojati značajna interakcija IAT-A i ponašanja
suigrača u prethodnim igrama, pri predviđanju donošenja nekooperativne odluke u
drugoj i trećoj igri. Značajna interakcija nije pronađena za drugu igru, ali
jest za treću. Ipak, ne radi se o jednostavnoj dvostrukoj interakciji ponašanja
suigrača i IAT-A, već o trostrukoj interakciji IAT-A, BPAQ\textsubscript{ukupno}
i odluke suigrača. Stoga, odbacit ćemo i Hipotezu 2.
Dok u slučaju prethodne nesuradnje postoje neke varijacije s
obzirom na razinu IAT-A i BPAQ\textsubscript{ukupno}, one nisu velike
te je moguće da niti nisu statistički značajne. U slučaju prethodne suradnje, 
visoka vjerojatnost nekooperativnog ponašanja predviđena je za sudionike s
profilima nekongruentnog poricanja i nekongruentnog preuveličavanja. Takav
obrazac jednak je onom trostruke interakcije opisane u prethodnom dijelu, u
situaciji kad je Dilema zatvorenika bila druga ili treća po redu.

Ponovno se pozivajući na Model Binga i sur. \citeyearpar{bing2007integrating},
možemo reći da su predviđanja vezana uz osobe profila nekongruentnog
poricanja u skladu s pretpostavkama. Kod donošenja presude o pojedincima s
profilom nekongruentnog preuveličavanja, muče nas iste poteškoće kao i pri
prethodnoj interpretaciji. Ovdje također nije jasno zbog čega bi predviđena
vjerojatnost donošenja nekooperativnih odluka za osobe s profilom kongruentne
prisutnosti bila niska.
Moguće je da osobe s takvim profilom zapravo
ne razvijaju opravdanja za svoje ponašanje, već da ga pokušavaju aktivno
potisnuti, budući da je socijalno nepoželjno, dok ga osobe s profilom
nekongruentnog poricanja zapravo nisu svjesne, pa se motivi u podlozi slobodno
izražavaju. 

Sama činjenica da ne postoji potpuno slaganje s Modelom ne mora biti
zabrinjavajuća. \citet{bing2007integrating} svoj su model testirali koristeći
testove uvjetnog rezoniranja, a kako su različite implicitne mjere pogodne za
predviđanje različitih kriterija \citep{uhlmann2012getting}, moguće je i da
zahvaćaju različite aspekte implicitne ličnosti te da se modeli temeljeni na
jednoj mjeri (ili jednoj paradigmi) ne mogu generalizirati na druge mjere.

Ipak, valja napomenuti da se ovaj model pokazao manje robusnim u replikacijama.
Odstupanja prosječnih koeficijenata od inicijalno procijenjenih veća su nego u
modelu opisanom u prošlom dijelu. Osim toga, tek oko trećina repliciranih
interakcija pokazala se statistički značajnima. Stoga, ovaj bi model valjalo
uzeti s većom dozom opreza. 

\subsection{Teorijske i praktične implikacije}

Ovo istraživanje je pokušaj dobivanja boljeg uvida u ljudsko donošenje odluka
i implicitne mjere ličnosti. Nažalost, nijedan očekivani odnos među varijablama
nije potvrđen. Ipak, značajnima su se pokazali neki odnosi koji nisu predviđeni
pri postavljanju istraživanja. Nalazi ukazuju na postojanje složenih interakcija
osobinskih faktora (implicitne i eksplicitne ličnosti) te situacijskih faktora
(ponašanje suigrača, vrsta eksperimentalne igre te širi kontekst igara).

Neki aspekti testiranih modela logističke regresije potvrđuju pretpostavke
Općeg modela prototipa ličnosti Binga i sur. \citeyearpar{bing2007integrating},
a neki ne. Bilo bi deplasirano propitivati valjanost Modela na temelju ovog
istraživanja, tim više što Model razmatra samo međuodnos implicitne i
eksplicitne komponente iste osobine ličnosti, ne uzimajući u obzir situacijske
faktore, koji su se u ovom, ali i prijašnjim istraživanjima
pokazali važnima za predviđanje nekooperativnih odgovora. 
Također, kao što je ranije spomenuto, moguće je da nepoklapanja proizlaze iz
korištenja različitih pristupa mjerenju.

S obzirom na kompleksnu prirodu međuodnosa varijabli, koja nije u potpunosti
razjašnjena, praktične implikacije su neznatne. U pogledu teorije, očito je da
je za adekvatno opisivanje međuodnosa potrebno provesti dodatna istraživanja. 
Proučavanje interakcije implicitnih i eksplicitnih mjera ličnosti te različitih
situacijskih varijabli vezanih uz eksperimentalne igre (prvenstveno ponašanja
suigrača) djeluje kao obećavajuć smjer. Variranje igara unutar jednog
istraživanja, s druge strane, ne djeluje obećavajuće, s obzirom na nepostojeće
korelacije između odluka u različitim igrama.

\subsection{Metodološka ograničenja}

Veličina uzorka predstavlja problem, zbog vjerojatno prerijetkog javljanja
nekooperativnih odluka s obzirom na broj prediktora --- \citet{orme2009multiple}
navode da se u literaturi mogu naći preporuke za od 5 do 16 opažanja po prediktoru za
rjeđu kategoriju kriterijske varijable. U našem istraživanju, tek su 23 sudionika
(39.22\%) donijela nekooperativnu odluku --- prema onom što navode
\citet{orme2009multiple}, trebalo bi ih biti barem 35. 

Također, iako su neka ranija istraživanja pokazala da se
odluke u nekim eksperimentalnim igrama ne razlikuju ovisno o tome jesu li
situacije stvarne ili hipotetske (npr. stvarno ili zamišljeno dodjeljivanje
novca u Igri diktatora, \citealp{ben2008economic}) te unatoč tome što se
u ovom istraživanju pokazalo da između sudionika koji su otkrili manipulaciju i
onih koji nisu ne postoji razlika u učestalosti donošenja nekooperativnih
odluka, ne može se u potpunosti isključiti mogućnost da je neuspješna obmana
utjecala na rezultate. Osim toga, moguće je da je dio sudionika bio upoznat s
teorijom igara te da odluke nisu donosili u potpunosti u skladu s vlastitim
motivima, već da su se orijentirali prema rješenjima iz teoretskog okvira. 
% vrsta implicitne mjere s obzirom na tip problema; možda jači efekt da je
% trajanje bilo vremenski ograničeno; inače je ovoj situaciji CRT možda bolji
% pristup? Getting implicit about the explicit

Nadalje, sudionicima je rečeno da će osobe s najvišim ukupnim rezultatom na
igrama moći osvojiti poklon bonove, no nije im objašnjeno na koji se način
odgovori boduju. Moguće je da bi rezultati bili drugačiji da je bodovanje
objašnjeno. Također, rezultati bi potencijalno bili drugačiji da su sudionici
dobivali kompenzaciju ovisno o učinku u zadacima.

Također, moguće je da sama eksperimentalna procedura nije odgovarala uvjetima u
kojima IAT postiže najbolju prediktivnost. \citet{uhlmann2012getting} IAT
svrstavaju u mjere temeljene na asocijacijama, za koje kažu da nisu prikladne za
procjenjivanje reakcija na podražaje koji zahtijevaju kompleksno procesiranje.
Stoga, potencijalno bolji pristup bio bi ograničiti vrijeme dostupno za
donošenje odluka u igrama ili čak koristiti neki drugi pristup mjerenju
implicitnih osobina ličnosti.

\section{Zaključak}

Cilj ovog istraživanja bio je proširiti postojeća znanja o bihevioralnoj teoriji
igara i testu implicitnih asocijacija za agresivnost. Očekivali smo da će biti
moguće predvidjeti donošenje nekooperativnih odluka u eksperimentalnim igrama na
temelju implicitnih i eksplicitnih mjera, njihove međusobne interakcije te
njihovih interakcija s ponašanjem suigrača. Hipoteze o prediktivnosti mjera i
njihovih interakcija nisu potvrđene. 

Dodatne analize --- koje su gledale trostruke
interakcije implicitne i eksplicitne mjere s ponašanjem suigrača odnosno
položajem igre u slijedu --- ukazuju na to da je međuodnos tih varijabli pri
predviđanju nekooperativnih odluka u eksperimentalnim igrama složeniji od
očekivanog. Naime, trostruka interakcija implicitne i eksplicitne mjere
s položajem igre u slijedu pokazala se značajnim prediktorom donošenja
nekooperativne odluke u Dilemi zatvorenika. Osim toga, trostruka interakcija
dviju korištenih mjera s ponašanjem suigrača pokazala se prediktivnom za
donošenje nekooperativnih odluka u trećoj igri po redu. 

Ipak, ove nalaze --- i značajne i neznačajne --- treba uzeti s oprezom.
Prikupljeni uzorak bio je razmjerno malen, vjerojatno premalen za otkrivanje
efekata, ukoliko su oni slabiji. Unatoč tome, ovi rezultati mogu poslužiti kao
grub putokaz za daljnja istraživanja na području implicitne ličnosti i
bihevioralne teorije igara.

\section{Zahvala}

Prvenstveno, želim zahvaliti svom mentoru Zvonimiru Galiću na vodstvu i podršci
koje je pružao prilikom izrade ovog rada. Želim zahvaliti i doktorandu Mitji
Ružojčiću, koji je uvijek bio dostupan za savjete i pitanja te pomagao oko
organiziranja tehničkih dijelova --- zbilja je bio izvrstan.
Želim zahvaliti Ivoni Eterović i Ozrenu Šiftaru jer su bili voljni
razgovarati o idejama koje su mi padale na pamet, što mi je uvelike
pomoglo pri definiranju ovog istraživanja. Zahvaljujem i Ivani Baran te, još
jednom, Ozrenu jer su preuzeli uloge procjenjivača pri provjeri
uspješnosti obmane. Ozrenu zahvaljujem i na ideji za način
ispitivanja uspješnosti obmane. Želim zahvaliti svojoj obitelji jer
mi je omogućila neopterećeno provođenje studentskih dana. Na kraju,
zahvaljujem svima koji su sudjelovali u mom istraživanju, kao i onima
koji su regrutirali sudionike. Kako ovaj popis nije potpun, a
njegovo upotpunjavanje zahtijevalo bi dokument od barem ovoliko
stranica, ukratko ću zahvaliti svima koji su imali ikakav pozitivan utjecaj na moj
život i obrazovanje.

\makeatletter
\addtocontents{toc}{\let\protect\l@chapter\protect\l@section}
\makeatother
\begin{appendices}

\subsection{Scenariji eksperimentalnih igara}

Scenarij Dileme zatvorenika\footnote{U tekstovima u programu nedostaju dijakritički
    znakovi jer nisu mogli biti adekvatno prikazani.}: 
\begin{quote}
Zamislite da se nalazite u sobi za ispitivanje u jednoj policijskoj postaji.  

Vi i Vaš kolega, koji je također na ispitivanju u istoj postaji, optuženi ste
za seriju pljački te vam prijeti nekoliko godina zatvora. 

Tužitelj Vam garantira da će biti blag prema Vama ako ćete svjedočiti protiv
drugoga. U slučaju da pristanete svjedočiti, a on ne, bit ćete pušteni, a on će
dobiti zatvorsku kaznu od 10 godina.

Ako obojica priznate, svaki će dobiti zatvorsku kaznu u trajanju od 3 godine.
Ako nijedan od vas ne prizna, svaki će dobiti zatvorsku kaznu u trajanju od 1
godine.

Tužitelj je istu nagodbu ponudio i Vašem kolegi. 

Što ćete učiniti? Hoćete li priznati i okriviti drugoga ili nećete ništa priznati?
\end{quote}

Ponuđene su opcije: (1) \enquote{ne priznajete zločin} i (2) \enquote{priznajete
    zločin i okrivljujete drugoga}. Tablica \ref{pdtab} prikazuje matricu ishoda
za Dilemu zatvorenika. 

\begin{table}[!h]
    \centering
    \caption{Matrica ishoda za Dilemu zatvorenika. \label{pdtab}}
\begin{tabular}{cccc}
\toprule[1pt]
    & & \multicolumn{2}{c}{\textcolor{plava}{Vaša odluka}}\\
    \cmidrule[0.5pt]{3-4}
        & & priznajete zločin & ne priznajete zločin\\
        \multirow{4}{*}{\textcolor{zelena}{Odluka Vašeg suigrača}} & \multirow{2}{*}{priznaje
        zločin} & \multicolumn{1}{r}{\textcolor{plava}{3 god.}} &
    \multicolumn{1}{r}{\textcolor{plava}{10 god.}}\\
                 & & \multicolumn{1}{l}{\textcolor{zelena}{3 god.}} &
                 \multicolumn{1}{l}{\textcolor{zelena}{0 god.}}\\
        \cmidrule[0.5pt]{2-4}
        & \multirow{2}{*}{ne priznaje zločin} &
        \multicolumn{1}{r}{\textcolor{plava}{0 god.}} &
        \multicolumn{1}{r}{\textcolor{plava}{1 god.}}\\
        & & \multicolumn{1}{l}{\textcolor{zelena}{10 god.}} &
        \multicolumn{1}{l}{\textcolor{zelena}{1 god.}}\\
	\bottomrule[1pt]
\end{tabular}
\end{table}

Scenarij Igre kukavice je: 
\begin{quote}
Zamislite da ste vozač jednog od dvaju automobila koji voze jedan prema drugom.
Ako nijedan od vas ne skrene, sudarit ćete se. Ukoliko jedan skrene, a drugi ne,
onaj koji je skrenuo bit će proglašen kukavicom. Ako obojica skrenete, obojica
ćete biti proglašeni kukavicama.

Što ćete učiniti? Hoćete li nastaviti voziti i riskirati sudar ili ćete skrenuti
i dobiti status kukavice, ali izbjeći prometnu nesreću?
\end{quote}

Ponuđene su opcije: (1) \enquote{skrenuti} i (2) \enquote{nastaviti voziti u
    istom smjeru}. Tablica \ref{iktab} prikazuje matricu ishoda
za Igru kukavice. 

\begin{table}[h!]
\centering 
    \caption{Matrica ishoda za Igru kukavice. \label{iktab}}
\hspace*{-0.5cm}\begin{tabular}{cccc}
\toprule[1pt]
    & & \multicolumn{2}{c}{Vaša odluka}\\
        & & nastavljate voziti & skrećete\\
    \cmidrule[0.5pt]{3-4}
        \multirow{2}{*}{Odluka Vašeg suigrača} & nastavlja
            voziti & sudarate se & ispali ste kukavica  \\
        & skreće & Vaš suigrač je ispao kukavica & oboje ste
        ispali kukavice\\
	\bottomrule[1pt]
\end{tabular}
\end{table}

Scenarij Igre vođe je: 
\begin{quote}
Zamislite da stojite na raskrižju i čekate priliku da se
uključite u gusti promet na glavnoj cesti. Pokraj Vas stoji još jedan
vozač, koji također čeka priliku za ulazak na glavnu cestu. 

Obojici vam se žuri i želite se što prije uključiti u promet.

Na glavnoj cesti se otvorio prostor koji jednom od vas omogućuje da se uključi
u promet. Ako obojica krenete istovremeno, riskirate sudar. S druge strane, ako
oboje ostanete stajati, propustit ćete priliku za priključivanje na glavnu
prometnicu.

Hoćete li se pokušati priključiti, i time riskirati sudaranje s autom u traci
pokraj, ili ćete dati prednost drugom vozilu, i riskirati da ćete oboje ostati
stajati i propustiti priliku?
\end{quote}

Ponuđene su opcije: (1) \enquote{dat ću prednost drugom vozilu} i (2)
\enquote{pokušat ću se priključiti u promet}. Tablica \ref{ivtab} prikazuje matricu ishoda
za Igru vođe. 

\begin{table}[!h]
\centering 
    \caption{Matrica ishoda za Igru vođe.\label{ivtab}}
\hspace*{-0.5cm}\begin{tabular}{cccc}
\toprule[1pt]
    & & \multicolumn{2}{c}{Vaša odluka}\\
    & & pokušavate se uključiti& čekate\\
    \cmidrule[0.5pt]{3-4}
    \multirow{2}{*}{Odluka Vašeg suigrača} & pokušava se uključiti& sudarate se
    & drugo vozilo se uključuje  \\
    & čeka & uključili ste se u promet & oboje ste ostali stajati\\
	\bottomrule[1pt]
\end{tabular}
\end{table}

\clearpage
\subsection{Dodatne analize}

\vspace*{\fill}
\begin{table}[h]
    \begin{center}
        \caption{\label{glmtotposP} Model logističke regresije s trostrukom interakcijom IAT-A,
            BPAQ\textsubscript{ukupno} i položajem Dileme zatvorenika u slijedu
            igara kao prediktorom
            nekooperativne odluke u Dilemi zatvorenika. 
            Za puni model: $\chi^2 =$ 9.274, $df =$ 7, $p =$ 0.23, $N
            =$ 102.}
        \hspace*{-0.7cm}\begin{tabular}{llrrrr}
        \toprule
        Model & Prediktor & $B$ & $SE$ & $z$ & OR (95\% CI)\\
        \midrule
        I & konstanta* & -1.923 & 0.690 & -2.787 &\\
        \multirow{3}{*}{$R^2_N =$ 0.02}
        &BPAQ\textsubscript{ukupno} & -0.066 & 0.296 & -0.221 &\\
        &IAT-A & -0.118 & 0.723 & 0.163 &\\
        &gameposP & 0.346 & 0.318 & 1.090  &\\
        &&&&&\\ 
        II & konstanta & -1.912 & 0.710 & -2.693 &\\
        \multirow{6}{*}{$R^2_N =$ 0.05}
        &BPAQ\textsubscript{ukupno} & 0.652 & 0.858 & 0.760 &\\
        &IAT-A & 0.652 & 2.118 & 0.308 &\\
        &gameposP & 0.303 & 0.329 & 0.922 &\\
        &BPAQ\textsubscript{ukupno} x IAT-A & -1.258 & 1.109 & -1.135 &\\
        &BPAQ\textsubscript{ukupno} x gameposP & -0.376 & 0.402 & -0.935 &\\
        &gameposP x IAT-A & -0.244 & 0.999 & -0.244 &\\
        &&&&&\\ 
        III & konstanta & -1.901 & 0.749 & -2.538 &\\
        \multirow{7}{*}{$R^2_N =$ 0.13}
        &BPAQ\textsubscript{ukupno} & 0.636 & 0.872 & 0.729 & \\
        &IAT-A & -0.296 & 2.320 & -0.128 &\\
        &gameposP & 0.211 & 0.364 & 0.579  &\\
        &BPAQ\textsubscript{ukupno} x IAT-A$^\ddagger$ & 5.799 & 3.198 & 1.813 &
        329.83 (0.751, 2.5 x 10\textsuperscript{5})\\
        &BPAQ\textsubscript{ukupno} x gameposP & -0.448 & 0.422 & -1.062 &\\
        &gameposP x IAT-A & 0.087 & 1.156 & 0.075 &\\
        & BPAQ\textsubscript{ukupno} x IAT-A x gameposP* & -3.815 &
        1.679 & -2.272 & 0.022 (0.001, 0.485)\\
        \bottomrule
        \multicolumn{6}{l}{
            \parbox{9cm}{\scriptsize \vspace{3pt} 
                * p < 0.05\\
                $\ddagger$ p < 0.07\\
                gameposP = položaj Dileme zatvorenika u slijedu igara
        }}
    \end{tabular}
\end{center}
\end{table}
\vspace*{\fill}
\clearpage

\begin{figure}
    \begin{center}
        \caption{\label{glmtotposPplot} Prikaz trostruke interakcije IAT-A,
            BPAQ\textsubscript{ukupno} i položaja igre kao prediktora
            nekooperativnog ponašanja u Dilemi zatvorenika. Kontinuirane
            varijable centrirane su na nulu (N = 102).}
        \includegraphics[keepaspectratio, width =
        \textwidth]{"/home/denis/Documents/Diplomski/diplomska radnja/diplomski.tex/glmagtotposp"}
    \end{center}
\end{figure}

\begin{table}
    \begin{center}
        \caption{\label{deskr p robust} Deskriptivni podaci replikacije punog
            modela s IAT-A, BPAQ\textsubscript{ukupno} i položajem Dileme
            zatvorenika kao prediktorima nekooperativnih odluka u Dilemi
            zatvorenika ($k =$ 1000).}
        \hspace*{-0.8cm}\begin{tabular}{lrrrrrrr}
        \toprule
        Prediktor & $M_B$ & $SD_B$ & $med_B$ & $IQR_B$ & $M_{SE}$ & $med_{SE}$
        & $\% p < 0.06$\\
        \midrule
        BPAQ\textsubscript{ukupno}$^\ddagger$ & 3.15 & 0.78 & 3.08 & 2.81 -- 3.43 &
        1.77 & 1.71 & 40.7\\
       IAT-A & -0.33 & 0.71 & -0.32 & -0.75 -- 0.07 & 2.50 & 2.47 & 0.0\\
       gameposP & 0.26 & 0.20 & 0.25 & 0.14 -- 0.36 & 0.72 & 0.70 & 0.0\\
       BPAQ\textsubscript{ukupno} x IAT-A$^\ddagger$ & 5.92 & 1.33 & 5.76 & 5.21 -- 6.52 &
        3.46 & 3.38 & 27.1\\
        BPAQ\textsubscript{ukupno} x gameposP* & -2.10 & 0.46 & -2.02 &
        -2.20 -- {-}1.88 & 0.96 & 0.92 & 89.2\\
       gameposP x IAT-A & 0.13 & 0.36 & 0.12 & -0.10 -- 0.31 & 1.25 & 1.22 & 0.0\\
       BPAQ\textsubscript{ukupno} x IAT-A x gameposP* & -3.88 & 0.76 & -3.76 &
        -4.15 -- {-}3.45 & 1.82 & 1.77 & 83.5\\
        \bottomrule
        \multicolumn{6}{l}{
            \parbox{9cm}{\scriptsize \vspace{3pt} 
                * p < 0.05\\
                $\ddagger$ p < 0.07\\
                gameposP = položaj Dileme zatvorenika u slijedu igara
        }}
    \end{tabular}
\end{center}
\end{table}

\begin{table}
    \begin{center}
        \caption{\label{glmtotgam3} Model logističke regresije s trostrukom interakcijom IAT-A,
            BPAQ\textsubscript{ukupno} i ponašanja suigrača kao prediktorom
            nekooperativne odluke u trećoj igri po redu. Prikazan je samo posljednji korak
            jer su raniji koraci prikazani u Tablici \ref{glmtotgam33}. Za puni
            model: $\chi^2 =$
            11.535, $df =$ 7, $p =$ 0.12, $N =$ 102.}
        \hspace*{-0.5cm}\begin{tabular}{llrrrr}
        \toprule
        Model & Prediktor & $B$ & $SE$ & $z$ & OR (95\% CI)\\
        \midrule
        III & konstanta* & -1.569 & 0.426 & -3.687 &\\
        \multirow{7}{*}{$R^2_N =$ 0.15}
        &BPAQ\textsubscript{ukupno} & -0.091 & 0.512 & -0.179 & \\
        &IAT-A & -0.622 & 1.387 & -0.448 &\\
        &botdec* & 1.101 & 0.518 & 2.128 & 3.006 (1.144, 9.083)\\
        &BPAQ\textsubscript{ukupno} x IAT-A* & -5.371 & 2.512 & -2.138 & 0.005
        (1.42 x 10\textsuperscript{-5}, 0.370) \\
        &BPAQ\textsubscript{ukupno} x botdec & -0.252 & 0.632 & -0.398 &\\
        &botdec x IAT-A & 0.707 & 1.703 & 0.415 &\\
        & BPAQ\textsubscript{ukupno} x IAT-A x botdec$^\ddagger$ & 5.112 & 2.804
        & 1.823 & 165.94 (1.019, 8328.1) \\
        \bottomrule
        \multicolumn{5}{l}{
            \parbox{3cm}{\scriptsize \vspace{3pt} 
                * p < 0.05\\
                $\ddagger$  p < 0.07\\
                botdec = odluka suigrača
        }}
    \end{tabular}
\end{center}
\end{table}

\begin{figure}
    \begin{center}
        \caption{\label{glmtotgam3plot} Prikaz trostruke interakcije IAT-A,
            BPAQ\textsubscript{ukupno} i ponašanja suigrača kao prediktora
            nekooperativnog ponašanja u trećoj igri po redu. Kontinuirane
            varijable centrirane su na nulu (N = 102).}
        \includegraphics[keepaspectratio, width =
        \textwidth]{"/home/denis/Documents/Diplomski/diplomska
            radnja/diplomski.tex/glmtot3"}
    \end{center}
\end{figure}

\begin{table}
    \begin{center}
        \caption{\label{deskr gam3 robust} Deskriptivni podaci replikacije punog
            modela s IAT-A, BPAQ\textsubscript{ukupno} i ponašanjem suigrača u
            prethodnim igrama kao prediktorima nekooperativnih odluka u trećoj 
            igri po redu ($k =$ 1000).}
        \hspace*{-0.8cm}\begin{tabular}{lrrrrrrr}
        \toprule
        Prediktor & $M_B$ & $SD_B$ & $med_B$ & $IQR_B$ & $M_{SE}$ & $med_{SE}$
        & $\% p < 0.07$\\
        \midrule
        BPAQ\textsubscript{ukupno}* & -2.61 & 1.18 & -2.26 & -2.52 -- -2.08 &
        1.34 & 1.20 & 65.9\\
       IAT-A & -0.81 & 1.32 & -0.53 & -0.77 -- -0.22& 1.54 & 1.41 & 02.1\\
       botdec & 1.56 & 0.89 & 1.33 & 1.13 -- 1.55 & 1.06 & 0.97 & 12.3\\
       BPAQ\textsubscript{ukupno} x IAT-A* & -5.77 & 2.30 & -5.16 & -5.85 --
       -4.74& 2.84 & 2.60 & 65.9\\
       BPAQ\textsubscript{ukupno} x botdec & 2.13 & 1.25 & 1.86 &
        1.57 -- 2.22 & 1.57 & 1.44 & 10.3\\
       botdec x IAT-A & 0.90 & 1.40 & 0.57 & 0.19 -- 1.05& 1.89 & 1.78 & 01.7\\
       BPAQ\textsubscript{ukupno} x IAT-A x botdec$^\ddagger$ & 5.48 & 2.39 & 4.94 &
        4.32 -- 5.82& 3.16 & 2.94 & 38.0\\
        \bottomrule
        \multicolumn{5}{l}{
            \parbox{3cm}{\scriptsize \vspace{3pt} 
                * p < 0.05\\
                $\ddagger$  p < 0.07\\
                botdec = odluka suigrača
        }}
    \end{tabular}
\end{center}
\end{table}

\end{appendices}

\bibliographystyle{apacite}

{
    \setstretch{1}
    \bibliography{diplomski.bib}
}

\end{document}
